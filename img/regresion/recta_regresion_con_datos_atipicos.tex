%% Input file name: recta_regresion_con_datos_atipicos.fig
%% FIG version: 3.2
%% Orientation: Landscape
%% Justification: Flush Left
%% Units: Inches
%% Paper size: A4
%% Magnification: 100.0
%% Resolution: 1200ppi
%% Include the following in the preamble:
%% \usepackage{textcomp}
%% End

\begin{pspicture}(7.40cm,3.29cm)(16.36cm,13.22cm)
\psset{unit=0.8cm}
%%
%% Depth: 2147483647
%%
\newrgbcolor{mycolor0}{1.00 0.50 0.31}\definecolor{mycolor0}{rgb}{1.00,0.50,0.31}
%%
%% Depth: 100
%%
\psset{linestyle=solid,linewidth=0.03175,linecolor=mycolor0}
\qdisk(18.79,7.12){0.1}
\qdisk(12.57,10.22){0.1}
\qdisk(17.26,7.66){0.1}
\qdisk(13.12,9.63){0.1}
\qdisk(11.56,10.56){0.1}
\qdisk(16.50,8.37){0.1}
\qdisk(15.06,8.73){0.1}
\qdisk(17.50,7.80){0.1}
\qdisk(11.97,10.53){0.1}
\qdisk(18.78,13.24){0.1}
\psset{linecolor=black,fillstyle=none}
\psline(11.37,6.32)(19.60,6.32)
\psline(11.37,6.32)(11.37,6.11)
\psline(13.02,6.32)(13.02,6.11)
\psline(14.66,6.32)(14.66,6.11)
\psline(16.31,6.32)(16.31,6.11)
\psline(17.95,6.32)(17.95,6.11)
\psline(19.60,6.32)(19.60,6.11)
\rput(11.37,5.56){0}
\rput(13.02,5.56){2}
\rput(14.66,5.56){4}
\rput(16.31,5.56){6}
\rput(17.95,5.56){8}
\rput(19.60,5.56){10}
\psline(11.04,6.65)(11.04,14.89)
\psline(11.04,6.65)(10.83,6.65)
\psline(11.04,8.30)(10.83,8.30)
\psline(11.04,9.95)(10.83,9.95)
\psline(11.04,11.59)(10.83,11.59)
\psline(11.04,13.24)(10.83,13.24)
\psline(11.04,14.89)(10.83,14.89)
\rput{90}(10.53,6.65){0}
\rput{90}(10.53,8.30){2}
\rput{90}(10.53,9.95){4}
\rput{90}(10.53,11.59){6}
\rput{90}(10.53,13.24){8}
\rput{90}(10.53,14.89){10}
\psline(11.04,6.32)(19.93,6.32)(19.93,15.21)(11.04,15.21)(11.04,6.32)
\rput(15.48,15.70){Recta de regresión con datos atípicos}
\rput[l](15.34,4.63){$X$}
\rput[l]{90}(9.77,10.63){$Y$}
\psset{linewidth=0.0635}
\psline(11.04,10.19)(19.93,8.52)
\rput(15.48,12.33){$y= -0.19 x + 4.21$}
\rput(15.48,11.89){$r^2 = 0.08$}
\end{pspicture}
%% End
