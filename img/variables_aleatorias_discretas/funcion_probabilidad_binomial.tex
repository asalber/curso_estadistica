%% Input file name: funcion_probabilidad_binomial.fig
%% FIG version: 3.2
%% Orientation: Landscape
%% Justification: Flush Left
%% Units: Inches
%% Paper size: A4
%% Magnification: 100.0
%% Resolution: 1200ppi
%% Include the following in the preamble:
%% \usepackage{textcomp}
%% End

\begin{pspicture}(5.45cm,3.48cm)(16.68cm,13.45cm)
\psset{unit=0.8cm}
%%
%% Depth: 2147483647
%%
\newrgbcolor{mycolor0}{1.00 0.50 0.31}\definecolor{mycolor0}{rgb}{1.00,0.50,0.31}
\newgray{mycolor1}{0.74}\definecolor{mycolor1}{gray}{0.74}
%%
%% Depth: 100
%%
\psset{linestyle=solid,linewidth=0.03175,linecolor=mycolor0}
\qdisk(10.61,6.83){0.1}
\qdisk(11.54,7.12){0.1}
\qdisk(12.47,8.23){0.1}
\qdisk(13.41,10.62){0.1}
\qdisk(14.34,13.49){0.1}
\qdisk(15.27,14.83){0.1}
\qdisk(16.21,13.49){0.1}
\qdisk(17.14,10.62){0.1}
\qdisk(18.07,8.23){0.1}
\qdisk(19.00,7.12){0.1}
\qdisk(19.94,6.83){0.1}
\psset{linecolor=black,fillstyle=none}
\psline(10.61,6.47)(19.94,6.47)
\psline(10.61,6.47)(10.61,6.26)
\psline(12.47,6.47)(12.47,6.26)
\psline(14.34,6.47)(14.34,6.26)
\psline(16.21,6.47)(16.21,6.26)
\psline(18.07,6.47)(18.07,6.26)
\psline(19.94,6.47)(19.94,6.26)
\rput(10.61,5.71){0}
\rput(12.47,5.71){2}
\rput(14.34,5.71){4}
\rput(16.21,5.71){6}
\rput(18.07,5.71){8}
\rput(19.94,5.71){10}
\psline(10.23,6.80)(10.23,14.95)
\psline(10.23,6.80)(10.02,6.80)
\psline(10.23,8.43)(10.02,8.43)
\psline(10.23,10.06)(10.02,10.06)
\psline(10.23,11.69)(10.02,11.69)
\psline(10.23,13.32)(10.02,13.32)
\psline(10.23,14.95)(10.02,14.95)
\rput{90}(9.73,6.80){0.00}
\rput{90}(9.73,8.43){0.05}
\rput{90}(9.73,10.06){0.10}
\rput{90}(9.73,11.69){0.15}
\rput{90}(9.73,13.32){0.20}
\rput{90}(9.73,14.95){0.25}
\psline(10.23,6.47)(20.31,6.47)(20.31,15.28)(10.23,15.28)(10.23,6.47)
\rput(15.27,15.99){Función de probabilidad de una binomial $B(10,0.5)$}
\rput(15.27,4.86){$X$}
\rput{90}(8.88,10.88){Probabilidad $f(x)$}
\psset{linecolor=mycolor1}
\psline(10.23,6.80)(20.31,6.80)
\end{pspicture}
%% End
