% Author: Alfredo Sánchez Alberca (email:asalber@ceu.es)
% Plot with the interquartile range
\begin{tikzpicture}
\draw (0,2) -- (5,2);
\draw[color=color1] (2.5,1.9) -- (2.5,2.1);
\node[text=color1] at (2.5,2.3) {$\bar x$};
\node[text=color1] at (2.5,1.7) {$5$};
\draw (0.5,1.9) -- (0.5,2.1);
\node at (0.5,1.7) {$1$}; 
\draw (4.5,1.9) -- (4.5,2.1);
\node at (4.5,1.7) {$9$};  
\node at (-1,2) {Student 1};
\node at (5.5,2) {$s=4$};
\pause
\draw (0,1) -- (5,1);
\draw[color=color1] (2.5,0.9) -- (2.5,1.1);
\node[text=color1] at (2.5,1.3) {$\bar x$};
\draw (2,0.9) -- (2,1.1);
\node at (2,0.7) {$4$}; 
\draw (3,0.9) -- (3,1.1);
\node at (3,0.7) {$6$};  
\node[text=color1] at (2.5,0.7) {$5$};
\node at (-1,1) {Student 2};
\node at (5.5,1) {$s=1$};
\end{tikzpicture}