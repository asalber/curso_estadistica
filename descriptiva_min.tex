%Version control information
%$HeadURL: https://curso-estadistica.googlecode.com/svn/trunk/descriptiva.tex $} {$LastChangedDate: 2011-12-05 12:17:24 +0100 (lun 05 de dic de 2011) $
%$LastChangedRevision: 15 $
%$LastChangedBy: asalber $

\title{Curso básico de Estadística\\
{\bfseries \emph{Construcción de tablas de frecuencias}}}
\author[]{Alfredo Sánchez Alberca (\texttt{\url{asalber@ceu.es}})} 
%\institute[USP CEU]{\includegraphics[scale=0.2]{img/logo_uspceu_01}}
\date{\textcopyleft Copyleft}

%---------------------------------------------------------------------slide----
\begin{frame}
\titlepage
\end{frame}


%---------------------------------------------------------------------slide----
\begin{frame}
\frametitle{Contenidos}
\videobox
\begin{enumerate}
\setbeamertemplate{enumerate item}{\insertenumlabel.}
\setbeamertemplate{enumerate subitem}{\insertenumlabel.\insertsubenumlabel}
\item Tipos de frecuencias
\item Construcción de tablas de frecuencias 
\begin{enumerate}
\item Datos sin agrupar
\item Datos agrupados
\item Atributos
\end{enumerate}
\end{enumerate}
\end{frame}


%---------------------------------------------------------------------slide----
\begin{frame}
\frametitle{Frecuencias muestrales}
\videobox
\vspace{1cm}
\begin{itemize}
\item \structure{Frecuencia absoluta $n_i$}:
\vspace{1cm} 
\item \structure{Frecuencia relativa $f_i$}: 
\vspace{1cm} 
\item \structure{Frecuencia absoluta acumulada $N_i$}:
\vspace{1cm}  
\item \structure{Frecuencia relativa acumulada $F_i$}:
\vspace{1cm}  
\end{itemize}
\note{Existen distintos tipos de frecuencias que pueden calcularse. 
Dada una muestra de tamaño $n$ de una variable $X$, para cada valor de la variable $x_i$ observado en la muestra, se define
\begin{itemize}
\item La frecuencia absoluta, que se representa como $n_i$, es el número de individuos de la
muestra que presentan el valor $x_i$, es decir el número de veces que se repite dicho valor.
\item La frecuencia relativa, que se denota $f_i$, es la proporción de individuos de
la muestra que presentan el valor $x_i$. La frecuencia relativa se calcula dividiendo la frecuencia absoluta entre el
tamaño de la muestra.
\[
f_i = \frac{n_i}{n}
\]
Cuando se multiplica por 100 se convierte en un porcentaje.
\item La frecuencia absoluta acumulada, que se escribe $N_i$, es el número de
individuos de la muestra que presentan un valor menor o igual que $x_i$. Se calcula acumulando las frecuencias
absolutas de los valores menores o iguales a $x_i$ y de ahí su nombre.
\[
N_i = n_1 + \cdots + n_i
\]
\item La frecuencia relativa acumulada, que se escribre $F_i$, es la proporción de
individuos de la muestra que presentan un valor menor o igual que $x_i$. Puede calcularse acumulando las frecuencias
relativas de los valores menores o iguales que $x_i$ o bien dividiendo la frecuencia absoluta cumulada de $x_i$ entre
el tamaño de la muestra.
\[
F_i = \frac{N_i}{n}
\]
Al igual que la frecuencia relativa, si se multiplica por 100 se convierte en el porcentaje acumulado.
\end{itemize}
 }
\end{frame}


%---------------------------------------------------------------------slide----
\begin{frame}
\frametitle{Tabla de frecuencias}
\videobox
\[
\setlength\arraycolsep{10mm}
\begin{array}{|c|c|c|c|c|}
\hline
 & & & & \\[1cm]
\hline
 & & & & \\[4cm]
\hline
\end{array}
\]
\note{Normalmente las frecuencias muestrales suelen organizarse en forma de tabla en la que cada fila corresponde a un
valor de la variable o un intervalo de valores, que se ordenan, siempre que sea posible, de menor a mayor, y cada
columna corresponde a una frecuencia.

En esta tabla siempre se debe cumplir que la suma de las frecuencias absolutas es igual al tamaño de la muestra, la
suma de las frecuencias relativas vale 1, la última frecuencia absoluta acumulada es el tamaño de la muestra y la
última frecuencia relativa acumulada es 1. De no ser así, habríamos cometido algún error en el cálculo de las
frecuencias.}
\end{frame}


%---------------------------------------------------------------------slide----
\begin{frame}
\frametitle{Tabla de frecuencias}
\framesubtitle{Ejemplo de datos sin agrupar}
\videobox
$X$= Número de hijos en una familia
\begin{center}
1, 2, 4, 2, 2, 2, 3, 2, 1, 1, 0, 2, 2, \\
 0, 2, 2, 1, 2, 2, 3, 1, 2, 2, 1, 2
\end{center}
Tabla de frecuencias
\[
\setlength\arraycolsep{5mm}
\setlength\arrayrulewidth{0.5pt}
\begin{array}{|c|c|c|c|c|}
\hline
x_i & n_i & f_i & N_i & F_i\\
\hline
 & & & & \\[3cm]

\hline 
 & & & & \\
\hline
\end{array}
\]
\note{En este ejemplo se han tomado 25 matrimonios en los que se ha medido el número de hijos que tenían. Se trata de
una variable cuantitativa discreta puesto que sólo puede tomar valores enteros positivos y además en la muestra sólo
aparece 5 valores disntintos que son 0, 1, 2, 3 y 4 hijos, por lo que no es necesario agrupar los datos en intervalos.

Para construir la tabla de frecuencias se comienza por el recuento de las frecuencias absolutas. Como puede observarse
en la muestra, hay dos matrimonios que tienen 0 hijos, y por tanto, la frecuencia absoluta del 0 es 2, el 1
aparece 6 veces, el 2, 3 veces, el 3, 2 veces y finalmente el 4 sólo aparece una vez. Observese cómo la suma de las
frecuencias absolutas da 25 que es el tamaño muestral.

A continuación se calculan las frecuencias relativas, simplemente dividiendo cada frecuencia absoluta por el tamaño de
la muestra que es 25. Por ejemplo, la frecuencia relativa del 0 es 2 entre 25 que es 0.08, y es la proporción de
matrimonios en la muestra que tienen 0 hijos.  Si se multiplica por 100 da un 8\%, es decir, el 8\% de los matrimonios
tienen 0 hijos. Obsérvese cómo la suma de las frecuencias relativas vale 1. 

Después se calculan las frecuencias absolutas acumuladas. Así, por ejemplo, la frecuencia absoluta acumulada del 0 es
el número de matrimonios que tienen 0 o menos hijos, y al ser el 0 el menor de los valores de la muestra, coincide con
la su frecuencia absoluta, que vale 2. La frecuencia absoluta acumulada del 1 es el número de matrimonios que tienen 1
o menos hijos, de manera que habría que acumula la frecuencia absoluta del 1 y del 0, es decir, 2 mas 6, que da un
total de 8, y así sucesivamente. En general para calcular cada frecuencia absoluta acumulada se puede tomar la
frecuencia absoluta acumulada anterior y sumarle la frecuencia absoluta del valor. 8+14=22, 22+2=24 y 24+1=25.
Obsérvese cómo la última frecuencia absoluta acumulada vale 25 que es el tamaño de la muestra.

Finalmente, se calculan las frecuencias relativas acumuladas, que pueden calcularse, bien acumulando las frecuencias
relativas del mismo modo que se acumulaban las absolutas, o bien dividiendo las frecuencias absolutas acumuladas por el
tamaño de la muestra. Así, por ejemplo, la frecuencia relativa acumulada del 0 es 2 entre 25, que vale 0.08 y coincide
con su frecuencia relativa. La frecuencia relativa acumulada del 1 es 8 entre 25 que vale 0.32 y es la proporción de
matrimonios en la muestra con 1 o menos hijos. Si se multiplica por 100 tenemos que hay un 32\% de matrimonios con 1 o
menos hijos. Y del mismo modo se calculan el resto de frecuencias relativas acumuladas, hasta llegar a la última que
siempre vale 1.
}
\end{frame}


%---------------------------------------------------------------------slide----
\begin{frame}
\frametitle{Tabla de frecuencias}
\framesubtitle{Ejemplo de datos agrupados}
\videobox
$X$= Estatura en cm
\begin{center}
179, 173, 181, 170, 158, 174, 172, 166, 194, 185,\\
162, 187, 198, 177, 178, 165, 154, 188, 166, 171,\\
175, 182, 167, 169, 172, 186, 172, 176, 168, 187.
\end{center}
Tabla de frecuencias
\[
\setlength\arraycolsep{5mm}
\setlength\arrayrulewidth{0.5pt}
\begin{array}{|c|c|c|c|c|}
\hline
x_i & n_i & f_i & N_i & F_i\\
\hline
 & & & & \\[3cm]

\hline 
 & & & & \\
\hline
\end{array}
\]
\note{En este otro ejemplo, se han medido las estaturas de 30 universitarios. Ahora se trata de una variable
cuantitativa continua, y como siempre ocurre con este tipo de variables, el número de valores disntintos que aparece en
la muestra suele ser demasiado grande, por lo que se tiene a agruparlos en intervalos.

En este caso se ha optado por construir 5 intervalos de amplitud 10 cm, empezando en 150 cm y terminando en 200 cm. 

El cálculo de frecuencias absolutas es similar al caso anterior, salvo que ahora no se cuenta el número de estaturas
que se repiten, sino el número de estaturas que caen en cada intervalo. Por ejemplo, la frecuencia
absoluta del intervalo $(150,160]$ es 2 ya que en la muestra hay dos personas, una que mide 158 y otra que mide 154, que caerían en este intervalo.
Una vez calculadas las frecuencias abolutas, el cálculo del resto de frecuencias es idéntico al caso de datos no
agrupados.}
\end{frame}


%---------------------------------------------------------------------slide----
\begin{frame}
\frametitle{Tabla de frecuencias}
\framesubtitle{Ejemplo con un atributo}
\videobox
$X$= Grupo sanguineo
\begin{center}
A, B, B, A, AB, 0, 0, A, B, B, A, A, A, A, AB,\\
A, A, A, B, 0, B, B, B, A, A, A, 0, A, AB, 0. 
\end{center}
\[
\setlength\arraycolsep{5mm}
\setlength\arrayrulewidth{0.5pt}
\begin{array}{|c|c|c|c|c|}
\hline
x_i & n_i & f_i & N_i & F_i\\
\hline
 & & & & \\[3cm]

\hline 
 & & & & \\
\hline
\end{array}
\]
\note{En este otro ejemplo, se ha medido el grupo sanguíneo de un grupo de 30 personas. Ahora se trata de un atributo
nominal, de manera que como no hay orden entre sus valores, puden ordenarse de cualquier manera en la tabla de
frecuencias, pero el cálculo de frecuencias se realiza como en casos anteriores, con la particularidad de que en este
caso no tiene sentido calcular las frecuencias acumuladas. ¿Te imaginas por qué?}
\end{frame}


%---------------------------------------------------------------------slide----
\begin{frame}
\frametitle{Sumario}
\videobox
\end{frame}