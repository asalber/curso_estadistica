\section{Teoría de la Probabilidad}

\mode<presentation>{
%---------------------------------------------------------------------slide----
\begin{frame}
\frametitle{Teoría de la  Probabilidad}
\tableofcontents[sectionstyle=show/hide,hideothersubsections]
\end{frame}
}


%---------------------------------------------------------------------slide----
\begin{frame}
\frametitle{Introducción}
La estadística descriptiva permite describir el comportamiento y las relaciones entre las variables en la muestra, pero no permite sacar conclusiones sobre el resto de la población.

Ha llegado el momento de dar el salto de la muestra a la población y pasar de la estadística descriptiva a la inferencia estadística, y el puente que lo permite es la \highlight{teoría de la probabilidad}.

Hay que tener en cuenta que el conocimiento que se puede obtener de la población a partir de la muestra es limitado, y que para obtener conclusiones válidas para la población la muestra debe ser representativa de esta.
Por esta razón, para garantizar la representatividad de la muestra, esta debe extraerse \emph{aleatoriamente}, es decir, al \emph{azar}.

La teoría de la probabilidad precisamente se encarga de controlar ese azar para saber hasta qué punto son fiables las conclusiones obtenidas a partir de una muestra.

\note{La estadística descriptiva permite describir el comportamiento y las relaciones entre las variables en la muestra, pero no permite sacar
conclusiones sobre el resto de la población.

Ha llegado el momento de dar el salto de la muestra a la población y pasar de la estadística descriptiva a la inferencia estadística, y el
puente que lo permite es la \structure{\textbf{teoría de la probabilidad}}.

Hay que tener en cuenta que el conocimiento que se puede obtener de la población a partir de la muestra es limitado, y que para obtener conclusiones válidas para la población la muestra debe ser representativa de esta.
Por esta razón, para garantizar la representatividad de la muestra, esta debe extraerse \emph{aleatoriamente}, es decir, al \emph{azar}.

Es aquí por tanto donde se necesita la teoría de la probabilidad, ya que esta precisamente se encarga de controlar ese azar para
saber hasta qué punto son fiables las conclusiones obtenidas a partir de una muestra.
}
\end{frame}


\subsection{Experimentos y sucesos aleatorios}

%---------------------------------------------------------------------slide----
\begin{frame}
\frametitle{Experimentos y sucesos aleatorios}
El estudio de una característica en una población se realiza a través de experimentos aleatorios.

\begin{definicion}[Experimento aleatorio] Un \emph{experimento aleatorio} es un experimento que cumple dos condiciones:
\begin{enumerate}
	\item El conjunto de posibles resultados es conocido.
	\item No se puede predecir con absoluta certeza el resultado del experimento.
\end{enumerate}
\end{definicion}

\textbf{Ejemplo}. Un ejemplo típico de experimentos aleatorios son los juegos de azar.
El lanzamiento de un dado, por ejemplo, es un experimento aleatorio ya que:
\begin{itemize}
\item Se conoce el conjunto posibles de resultados $\{1,2,3,4,5,6\}$.
\item Antes de lanzar el dado, es imposible predecir con absoluta certeza el valor que saldrá. 
\end{itemize}

Otro ejemplo de experimento aleatorio sería la selección de un individuo de una población al azar y la determinación de su grupo sanguíneo.

En general, la obtención de cualquier muestra mediante procedimientos aleatorios será un experimento aleatorio.

\note{El estudio de una característica en una población se realiza a través de experimentos aleatorios.

\begin{definicion}[Experimento aleatorio] Un \emph{experimento aleatorio} es un experimento que cumple dos condiciones:
\begin{enumerate}
	\item El conjunto de posibles resultados es conocido.
	\item No se puede predecir con absoluta certeza el resultado del experimento.
\end{enumerate}
\end{definicion}

Un ejemplo sencillo de experimentos aleatorios son los juegos de azar. Por ejemplo, el lanzamiento de un dado es un experimento aleatorio ya
que:
\begin{itemize}
\item Se conoce el conjunto posibles de resultados $\{1,2,3,4,5,6\}$.
\item Antes de lanzar el dado, es imposible predecir con absoluta certeza el valor que saldrá. 
\end{itemize}

A lo largo del tema utilizaremos varias veces los juegos de azar para dar ejemplos sencillos que ayuden a comprender los conceptos de la
teoría de la probabilidad, pero también daremos ejemplos más aplicados a las ciencias de la salud. 

Otro ejemplo de experimento aleatorio más aplicado a la salud sería la selección de un individuo de una población al azar y la determinación
de su grupo sanguíneo. En este caso el conjunto de posibles resultados serían los cuatro grupos sanguíneos A, B, 0 y AB. De nuevo antes de
seleccionar la individuo es imposible predecir con absoluta certeza cuál será el grupo sanguíneo resultante. 

En general, la obtención de cualquier muestra mediante procedimientos aleatorios será un experimento aleatorio.  
}
\end{frame}


%---------------------------------------------------------------------slide----
\begin{frame}
\frametitle{Espacio muestral}
\begin{definicion}[Espacio muestral]
Al conjunto $\Omega$ de todos los posibles resultados de un experimento aleatorio se le llama \emph{espacio muestral}.
\end{definicion}

Algunos ejemplos de espacios muestrales son:
\begin{itemize}
\item Lanzamiento de una moneda: $\Omega=\{c,x\}$.
\item Lanzamiento de un dado: $\Omega=\{1,2,3,4,5,6\}$.
\item Grupo sanguíneo de un individuo seleccionado al azar: $\Omega=\{\mbox{A},\mbox{B},\mbox{AB},\mbox{0}\}$.
\item Estatura de un individuo seleccionado al azar: $\Omega=\mathbb{R}^+$.
\end{itemize}

\note{Al conjunto $\Omega$ de todos los posibles resultados de un experimento aleatorio se le llama \emph{espacio muestral}.
\end{definicion}

Algunos ejemplos de espacios muestrales son:
\begin{itemize}
\item Lanzamiento de una moneda: $\Omega=\{c,x\}$.
\item Lanzamiento de un dado: $\Omega=\{1,2,3,4,5,6\}$.
\item Grupo sanguíneo de un individuo seleccionado al azar: $\Omega=\{\mbox{A},\mbox{B},\mbox{AB},\mbox{0}\}$.
\item Estatura de un individuo seleccionado al azar: $\Omega=\mathbb{R}^+$. 
En este último caso se trataría de un conjunto con infinitos valores. 
\end{itemize}
}
\end{frame}


%---------------------------------------------------------------------slide----
\begin{frame}
\frametitle{Diagrama de árbol}
En experimentos donde se mide más de una variable, la determinación del espacio muestral puede resultar compleja. 
En tales casos es recomendable utilizar un \highlight{diagrama de árbol} para construir el espacio muestral.
	
En un diagrama de árbol cada variable se representa en un nivel del árbol y cada posible valor de la variable como una rama. 
\end{frame}
	
	
%---------------------------------------------------------------------slide----
\begin{frame}
\frametitle{Diagrama de árbol}
\framesubtitle{Ejemplo de sexo y grupo sanguíneo}
	
\begin{center}
\tikzsetnextfilename{probabilidad/espacio_muestral}
\mode<article>{\resizebox{0.6\textwidth}{!}{% Author: Alfredo Sánchez Alberca (asalber@ceu.es)

\begin{tikzpicture}[
grow'=right,
% sloped,
level 1/.style ={level distance=2cm, sibling distance=3.2cm, parent anchor=east, child anchor=west},
level 2/.style ={level distance=4cm, sibling distance=0.8cm},
level 3/.style ={level distance=2cm, sibling distance=0.8cm, dashed}
]

\node (root) {}
	child {
		node {Mujer}
    	child {node {0}
				child {node{(Mujer,0)}}
			}
 			child {node {A}
				child {node{(Mujer,A)}}
			}
			child {node {B}
				child {node{(Mujer,B)}}
			}
			child {node {AB}
				child {node{(Mujer,AB)}}
			}
    }
	child {
		node {Hombre}
    	child {node {0}
				child {node{(Hombre,0)}}
			}
 			child {node {A}
				child {node{(Hombre,A)}}
			}
			child {node {B}
				child {node{(Hombre,B)}}
			}
			child {node {AB}
				child {node{(Hombre,AB)}}
			}
    };

\begin{scope}[every node/.style={text width=2cm, align=center, anchor=center, font=\bfseries,}]
\node[above= 0.5cm of root-1-1-1] (labels-level) {$\Omega$};
\node[at =(labels-level-|root-1-1)] {Grupo sanguíneo};
\node[at =(labels-level-|root-1)] {Sexo};
\end{scope}
\end{tikzpicture}

}}
\mode<presentation>{\resizebox{0.6\textwidth}{!}{% Author: Alfredo Sánchez Alberca (asalber@ceu.es)

\begin{tikzpicture}[
grow'=right,
% sloped,
level 1/.style ={level distance=2cm, sibling distance=3.2cm, parent anchor=east, child anchor=west},
level 2/.style ={level distance=4cm, sibling distance=0.8cm},
level 3/.style ={level distance=2cm, sibling distance=0.8cm, dashed}
]

\node (root) {}
	child {
		node {Mujer}
    	child {node {0}
				child {node{(Mujer,0)}}
			}
 			child {node {A}
				child {node{(Mujer,A)}}
			}
			child {node {B}
				child {node{(Mujer,B)}}
			}
			child {node {AB}
				child {node{(Mujer,AB)}}
			}
    }
	child {
		node {Hombre}
    	child {node {0}
				child {node{(Hombre,0)}}
			}
 			child {node {A}
				child {node{(Hombre,A)}}
			}
			child {node {B}
				child {node{(Hombre,B)}}
			}
			child {node {AB}
				child {node{(Hombre,AB)}}
			}
    };

\begin{scope}[every node/.style={text width=2cm, align=center, anchor=center, font=\bfseries,}]
\node[above= 0.5cm of root-1-1-1] (labels-level) {$\Omega$};
\node[at =(labels-level-|root-1-1)] {Grupo sanguíneo};
\node[at =(labels-level-|root-1)] {Sexo};
\end{scope}
\end{tikzpicture}

}}
\end{center}
\end{frame}


%---------------------------------------------------------------------slide----
\begin{frame}
\frametitle{Sucesos aleatorios}
\begin{definicion}[Suceso aleatorio]
Un \emph{suceso aleatorio} es cualquier subconjunto del espacio muestral $\Omega$ de un experimento aleatorio. 
\end{definicion}

Existen distintos tipos de sucesos:
\begin{description}
	\item[Suceso imposible:] Es el suceso vacío $\emptyset$. Este suceso nunca ocurre.
	\item[Sucesos elementales:] Son los sucesos formados por un solo elemento.
	\item[Sucesos compuestos:] Son los sucesos formados por dos o más elementos.
	\item[Suceso seguro:] Es el suceso que contiene el propio espacio muestral $\Omega$. Este suceso siempre ocurre.
\end{description}

\note{
\begin{definicion}
Un \emph{suceso aleatorio} es cualquier subconjunto del espacio muestral $\Omega$ de un experimento aleatorio. 
\end{definicion}

Existen distintos tipos de sucesos:
\begin{description}
\item[Suceso imposible:] Es el subconjunto vacío $\emptyset$. El suceso nunca ocurre. Se llama imposible porque al realizar el experimento,
no se sabe qué ocurrirá pero es imposible que no ocurra nada. 
\item[Sucesos elementales:] Son los subconjuntos formados por un solo elemento. Hay tantos sucesos elementales como elementos tenga el
espacio muestral.
\item[Sucesos compuestos:] Son los subconjuntos formados por dos o más elementos.
\item[Suceso seguro:] Es el propio espacio muestral. Se llama suceso seguro porque al realizar el experimento siempre ocurrirá algo del
espacio muestral. 
\end{description}
}
\end{frame}


\subsection{Teoría de conjuntos}

%---------------------------------------------------------------------slide----
\begin{frame}
\frametitle{Espacio de sucesos}
\begin{definicion}[Espacio de sucesos] Dado un espacio muestral $\Omega$ de un
experimento aleatorio, el conjunto formado por todos los posibles sucesos de $\Omega$ se llama \emph{espacio de sucesos de $\Omega$} y se denota
$\mathcal{P}(\Omega)$.
\end{definicion}

Ejemplo. Dado el espacio muestral $\Omega=\{a,b,c\}$, su espacio de sucesos es
\[
\mathcal{P}(\Omega)=\left\{\emptyset, \{a\},\{b\},\{c\},\{a,b\},\{a,c\},\{b,c\},\{a,b,c\}\right\}
\]

\note{
\begin{definicion}[Espacio de sucesos] Dado un espacio muestral $\Omega$ de un
experimento aleatorio, el conjunto formado por todos los posibles sucesos de $\Omega$ se llama \emph{espacio de sucesos de $\Omega$} y se denota
$\mathcal{P}(\Omega)$, también conocido como partes de $\Omega$.
\end{definicion}

\textbf{Ejemplo}. Dado el espacio muestral $\Omega=\{a,b,c\}$, su espacio de sucesos es
\[
\mathcal{P}(\Omega)=\left\{\emptyset, \{a\},\{b\},\{c\},\{a,b\},\{a,c\},\{b,c\},\{a,b,c\}\right\}
\]
}
\end{frame}


%---------------------------------------------------------------------slide----
\begin{frame}
\frametitle{Operaciones entre sucesos}
Puesto que los sucesos son conjuntos, por medio de la teoría de conjuntos se pueden definir las siguientes operaciones entre sucesos:
\begin{itemize}
\item Unión.
\item Intersección.
\item Complementario.
\item Diferencia.
\end{itemize}

\note{
Puesto que los sucesos son conjuntos, tiene sentido definir operaciones entre sucesos a partir de la teoría de conjuntos. Las operaciones
más habituales que se verán son: 
\begin{itemize}
\item Unión.
\item Intersección.
\item Complementario.
\item Diferencia.
\end{itemize}
}
\end{frame}


% ---------------------------------------------------------------------slide----
\begin{frame}
\frametitle{Unión de sucesos}
\begin{definicion}[Suceso unión]
Dados dos sucesos $A,B\subseteq \Omega$, se llama \emph{suceso unión} de $A$ y $B$, y se denota $A\cup B$, al suceso formado por los elementos de $A$ junto a los elementos de $B$, es decir,
\[
A\cup B = \{x\,|\, x\in A\textrm{ o }x\in B\}.
\]
\end{definicion}

\begin{center}
	\tikzsetnextfilename{probabilidad/union}
	\input{img/probabilidad/union}
\end{center}

El suceso unión $A\cup B$ ocurre siempre que ocurre $A$ \alert{o} $B$.

\note{
\begin{definicion}[Suceso unión]
Dados dos sucesos $A,B\subseteq \Omega$, se llama \emph{suceso unión} de $A$ y $B$, y se denota $A\cup B$, al suceso formado por los
elementos de $A$ junto a los elementos de $B$, es decir,
\[
A\cup B = \{x\,|\, x\in A\textrm{ o }x\in B\}.
\]
\end{definicion}

El suceso unión $A\cup B$ ocurre siempre que ocurre $A$ \alert{o} $B$ y por tanto está asociado al nexo o del castellano.

Ejemplo. Sea $\Omega=\{1,2,3,4,5,6\}$, el conjunto de los números de un dado, y $A=\{2,4,6\}$ y $B=\{1,2,3,4\}$. Entonces $A\cup
B=\{1,2,3,4,6\}$.
}
\end{frame}


%---------------------------------------------------------------------slide----
\begin{frame}
\frametitle{Intersección de sucesos}
\begin{definicion}[Suceso intersección]
Dados dos sucesos $A,B\subseteq \Omega$, se llama \emph{suceso intersección} de $A$ y $B$, y se denota $A\cap B$, al suceso formado por los elementos comunes de $A$ y $B$, es decir,
\[
A\cap B = \{x\,|\, x\in A\textrm{ y }x\in B\}.
\]
\end{definicion}

\begin{center}
	\tikzsetnextfilename{probabilidad/interseccion}
	\input{img/probabilidad/interseccion}
\end{center}

El suceso intersección $A\cap B$ ocurre siempre que ocurren $A$ \alert{y} $B$.

Diremos que dos sucesos son \structure{\textbf{incompatibles}} si su intersección es vacía.

\note{
\begin{definicion}[Suceso intersección]
Dados dos sucesos $A,B\subseteq \Omega$, se llama \emph{suceso intersección} de $A$ y $B$, y se denota $A\cap B$, al suceso formado por los elementos comunes de $A$ y $B$, es decir,
\[
A\cap B = \{x\,|\, x\in A\textrm{ y }x\in B\}.
\]
\end{definicion}

El suceso intersección $A\cap B$ ocurre siempre que ocurren $A$ \alert{y} $B$, y por tanto está asociado al nexo y del castellano.

Diremos que dos sucesos son \structure{\textbf{incompatibles}} cuando no tienen nada en común, es decir, si su intersección es vacía.
}
\end{frame}


%---------------------------------------------------------------------slide----
\begin{frame}
\frametitle{Contrario de un suceso}
\begin{definicion}[Suceso contrario]
Dado suceso $A\subseteq \Omega$, se llama \emph{suceso contrario o complementario} de $A$, y se denota $\overline A$, al suceso formado por los elementos de $\Omega$ que no pertenecen a $A$, es decir,
\[
\overline A = \{x\,|\, x\not\in A\}.
\]
\end{definicion}

\begin{center}
	\tikzsetnextfilename{probabilidad/contrario}
	\input{img/probabilidad/contrario}
\end{center}

El suceso contrario $\overline A$ ocurre siempre que \alert{no} ocurre $A$.


\note{
\begin{definicion}[Suceso contrario]
Dado un suceso $A\subseteq \Omega$, se llama \emph{suceso contrario o complementario} de $A$, y se denota $\overline A$, al suceso formado por los elementos de $\Omega$ que no pertenecen a $A$, es decir,
\[
\overline A = \{x\,|\, x\not\in A\}.
\]
\end{definicion}

El suceso contrario $\overline A$ ocurre siempre que \alert{no} ocurre $A$, y por tanto esta ligado a la negación en castellano. 
}
\end{frame}


%---------------------------------------------------------------------slide----
\begin{frame}
\frametitle{Diferencia de sucesos}
\begin{definicion}[Suceso diferencia]
Dados dos sucesos $A,B\subseteq \Omega$, se llama \emph{suceso diferencia} de $A$ y $B$, y se denota $A-B$, al suceso formado por los elementos de $A$ que no pertenecen a $B$, es decir,
\[
	A-B = \{x\,|\, x\in A\mbox{ y }x\not\in B\} = A \cap \overline B.
\]
\end{definicion}

\begin{center}
	\tikzsetnextfilename{probabilidad/diferencia}
	\input{img/probabilidad/diferencia}
\end{center}

El suceso diferencia $A-B$ ocurre siempre que ocurre $A$ pero no ocurre $B$, y también puede expresarse como $A\cap \bar B$.

\note{
\begin{definicion}[Suceso diferencia]
Dados dos sucesos $A,B\in \mathcal{P}(\Omega)$, se llama \emph{suceso diferencia} de $A$ y $B$, y se denota $A-B$, al suceso formado por los elementos de $A$ que no pertenecen a $B$, es decir,
\[
A-B = \{x\,|\, x\in A\textrm{ y }x\not\in B\}.
\]
\end{definicion}

El suceso diferencia $A-B$ ocurre siempre que ocurre $A$ pero no ocurre $B$, y también puede expresarse como $A\cap \bar B$.
}
\end{frame}


%---------------------------------------------------------------------slide----
\begin{frame}
\frametitle{Operaciones entre sucesos}
\framesubtitle{Ejemplo}
Dado el espacio muestral correspondiente al lanzamiento de un dado $\Omega=\{1,2,3,4,5,6\}$ y los sucesos $A=\{2,4,6\}$ y $B=\{1,2,3,4\}$, 
\begin{itemize}
	\item La unión de $A$ y $B$ es $A\cup B=\{1,2,3,4,6\}$.
	\item La intersección de $A$ y $B$ es $A\cap B=\{2,4\}$.
	\item El contrario de $A$ es $\overline A=\{1,3,5\}$.
	\item Los eventos $A$ y $\overline A$ son incompatibles.
	\item La diferencia de $A$ y $B$ es $A-B=\{6\}$, y la diferencia de $B$ y $A$ es $B-A=\{1,3\}$.
\end{itemize}
\end{frame}


%---------------------------------------------------------------------slide----
\begin{frame}
\frametitle{Álgebra de sucesos}
Dados los sucesos $A,B,C\in  \mathcal{P}(\Omega)$, se cumplen las siguientes propiedades:
\begin{enumerate}
\item $A\cup A=A$, $A\cap A=A$ (idempotencia).
\item $A\cup B=B\cup A$, $A\cap B = B\cap A$ (conmutativa).
\item $(A\cup B)\cup C = A\cup (B\cup C)$, $(A\cap B)\cap C = A\cap (B\cap C)$ (asociativa).
\item $(A\cup B)\cap C = (A\cap C)\cup (B\cap C)$, $(A\cap B)\cup C = (A\cup C)\cap (B\cup C)$ (distributiva).
\item $A\cup \emptyset=A$, $A\cap E=A$ (elemento neutro).
\item $A\cup E=E$, $A\cap \emptyset=\emptyset$ (elemento absorbente).
\item $A\cup \overline A = E$, $A\cap \overline A= \emptyset$ (elemento simétrico complementario).
\item $\overline{\overline A} = A$ (doble contrario).
\item $\overline{A\cup B} = \overline A\cap \overline B$, $\overline{A\cap B} = \overline A\cup \overline B$ (leyes de Morgan).
\item $A\cap B\subseteq A\cup B$.
\end{enumerate}

\note{
Las operaciones entre sucesos vistas cumplen toda una serie de propiedades que no demostraremos aunque algunas de ellas son fáciles de
comprobar:
\begin{enumerate}
\item $A\cup A=A$, $A\cap A=A$ (idempotencia).
\item $A\cup B=B\cup A$, $A\cap B = B\cap A$ (conmutativa).
\item $(A\cup B)\cup C = A\cup (B\cup C)$, $(A\cap B)\cap C = A\cap (B\cap C)$ (asociativa).
\item $(A\cup B)\cap C = (A\cap C)\cup (B\cap C)$, $(A\cap B)\cup C = (A\cup C)\cap (B\cup C)$ (distributiva).
\item $A\cup \emptyset=A$, $A\cap E=A$ (elemento neutro).
\item $A\cup E=E$, $A\cap \emptyset=\emptyset$ (elemento absorbente).
\item $A\cup \overline A = E$, $A\cap \overline A= \emptyset$ (elemento simétrico complementario).
\item $\overline{\overline A} = A$ (doble contrario).
\item $\overline{A\cup B} = \overline A\cap \overline B$, $\overline{A\cap B} = \overline A\cup \overline B$ (leyes de Morgan).
\item $A\cap B\subseteq A\cup B$.
\end{enumerate} 
}
\end{frame}


\subsection{Definición de probabilidad}

%---------------------------------------------------------------------slide----
\begin{frame}
\frametitle{Definición clásica de probabilidad}
\begin{definicion}[Probabilidad --- Laplace]
Dado un espacio muestral $\Omega$ de un experimento aleatorio donde todos los elementos de $\Omega$ son equiprobables, la \emph{probabilidad} de un suceso $A\subseteq \Omega$ es el cociente entre el número de elementos de $A$ y el número de elementos de $\Omega$
\[ 
	P(A) = \frac{|A|}{|\Omega|} = \frac{\mbox{nº casos favorables a A}}{\mbox{nº casos posibles}}
\]
\end{definicion}

Esta definición es ampliamente utilizada, aunque tiene importantes restricciones:
\begin{itemize}
\item Es necesario que todos los elementos del espacio muestral tengan la misma probabilidad de ocurrir (\emph{equiprobabilidad}). 
\item No puede utilizarse con espacios muestrales infinitos, o de los que no se conoce el número de casos posibles. 
\end{itemize}

\alert{\emph{¡Ojo! Esto no se cumple en muchos experimentos aleatorios reales.}}

\note{
Como hemos visto, en u experimento aleatorio, un determinado suceso puede ocurrir o no al realizar el experimento, es decir, existe cierta
incertidumbre sobre su ocurrencia. Tiene sentido, por tanto, cuantificar dicha incertidumbre y medir así la verosimilitud del suceso o la
confianza que se tiene en que ocurra, y esto es lo que trata de hacer la probabilidad.

La definición clásica de probabilidad se debe al matemático francés Laplace y define la probabilidad de un suceso como un cociente entre el
número de elementos del suceso y el número de elementos del espacio muestral del experimento aleatorio, más conocida como casos favorables
al suceso entre casos posibles del espacio muestral.

Esta definición es ampliamente utilizada, aunque tiene importantes restricciones:
\begin{itemize}
\item No puede utilizarse con espacios muestrales infinitos, o de los que no se conoce el número de casos posibles.
\item Es necesario que todos los elementos del espacio muestral tengan la misma probabilidad de ocurrir (\emph{equiprobabilidad}).  
\end{itemize}
Aunque esto suele cumplirse en los juegos de azar, en otros muchos experimentos no es cierto. Por ejemplo los grupos sanguíneos no son
equiprobables ya que hay mucha más gente con el grupo $A$ que con otro grupo. 
}
\end{frame}


%---------------------------------------------------------------------slide----
\begin{frame}
\frametitle{Definición frecuentista de probabilidad}
\begin{teorema}[Ley de los grandes números]
Cuando un experimento aleatorio se repite un gran número de veces, las frecuencias relativas de los sucesos del experimento tienden a estabilizarse en torno a cierto número, que es precisamente su probabilidad.
\end{teorema}

De acuerdo al teorema anterior, podemos dar la siguiente definición
\begin{definicion}[Probabilidad frecuentista]
Dado un espacio muestral $\Omega$ de un experimento aleatorio reproducible, la \emph{probabilidad} de un suceso $A\subseteq \Omega$ es la frecuencia relativa del suceso $A$ en infinitas repeticiones del experimento
\[ 
	P(A) = lim_{n\rightarrow \infty}\frac{n_{A}}{n}
\]
\end{definicion}

Aunque esta definición es muy útil en experimentos científicos reproducibles, también tiene serios inconvenientes, ya que
\begin{itemize}
\item Sólo se calcula una aproximación de la probabilidad real.
\item La repetición del experimento debe ser en las mismas condiciones.  
\end{itemize}

\note{
Otra definición común de probabilidad surge de la ley de los grandes números que dice que 
cuando un experimento aleatorio se repite un gran número de veces, las frecuencias relativas de los sucesos del
experimento tienden a estabilizarse en torno a cierto número, que es precisamente su probabilidad.

Un ejemplo que demuestra el cumplimiento de esta ley puede realizarse tirando múltiples veces una moneda y anotando la frecuencia relativa
de caras. A medida que se tire más veces la moneda se verá que la frecuencia relativa de caras se va estabilizando en torno a $0.5$ que es
la probabilidad de sacar cara. 

Para un experimento aleatorio reproducible, se puede definir así la probabilidad de un suceso como el límite cuando el número de
repeticiones del mismo tiende a infinito de la frecuencia relativa del suceso.

Esta definición permitiría calcular de manera aproximada la probabilidad de que una persona elegida al azar tenga grupo sanguíneo $A$
simplemente tomando una muestra grande de la población y calculando la frecuencia relativa de personas con el grupo $A$ en la muestra.
Cuanto mayor sea la muestra, mejor será la aproximación de la verdadera probabilidad del grupo $A$.

Aunque esta definición es muy útil en experimentos científicos reproducibles, también tiene serios inconvenientes, ya que
\begin{itemize}
\item Sólo se calcula una aproximación de la probabilidad real.
\item La repetición del experimento debe ser en las mismas condiciones.  
\end{itemize}
}
\end{frame}


%---------------------------------------------------------------------slide----
\begin{frame}
\frametitle{Definición axiomática de probabilidad}
\begin{definicion}[Kolmogórov]
Dado un espacio muestral $\Omega$ de un experimento aleatorio, una función de \emph{probabilidad} es una aplicación que asocia a cada suceso $A\subseteq \Omega$ un número real $P(A)$, conocido como probabilidad de $A$, que cumple los siguientes axiomas:
\begin{enumerate}
\item La probabilidad de un suceso cualquiera es positiva o nula,
\[
	P(A)\geq 0.
\]
\item La probabilidad del suceso seguro es igual a la unidad,
\[
	P(\Omega)=1.
\]
\item La probabilidad de la unión de dos sucesos incompatibles ($A\cap B=\emptyset$) es igual a la suma de las probabilidades de cada uno de ellos,
\[
	P(A\cup B) = P(A)+P(B).
\]
\end{enumerate}
\end{definicion}

\note{
La definición de probabilidad más aceptada actualmente es la definición axiomática de Kolmogórov, quien definió la probabilidad como 
una medida de verosimilitud que asocia a cada suceso de un experimento aleatorio un número real, conocido como probabilidad del suceso, y
que cumple tres axiomas:
\begin{enumerate}
\item La probabilidad de un suceso cualquiera es positiva o nula: \[P(A)\geq 0.\]
\item La probabilidad de la unión de dos sucesos incompatibles es igual a la suma de las probabilidades de cada uno de ellos:       
\[P(A\cup B) = P(A)+P(B).\]
\item La probabilidad del suceso seguro es igual a la unidad: 
\[P(E)=1.\]
\end{enumerate}

Estos axiomas los cumplen todas las definiciones históricas de probabilidad y por tanto, todas ellas tienen cabida bajo la definición
axiomática de Kolmogórov.
}
\end{frame}


%---------------------------------------------------------------------slide----
\begin{frame}
\frametitle{Consecuencias de los axiomas de probabilidad}
A partir de los axiomas de la definición de probabilidad se pueden deducir los siguientes resultados:
\begin{enumerate}
\item <2-> $P(\overline A) = 1-P(A)$.
\item <3->$P(\emptyset)= 0$.
\item <4->Si $A\subseteq B$ entonces $P(A)\leq P(B)$.
\item <5->$P(A) \leq 1$.
\item <6->Si $A$ y $B$ son sucesos compatibles, es decir, su intersección no es vacía, entonces 
\[P(A\cup B)= P(A) + P(B) - P(A\cap B).\]
\item <7->Si el suceso $A$ está compuesto por los sucesos elementales $e_1,e_2,...,e_n$, entonces
\[P(A)=\sum_{i=1}^n P(e_i).\]
\end{enumerate}

\mode<article>{
	\textbf{Demostración}
	\begin{enumerate}
	\item $\overline A = \Omega \Rightarrow P(A\cup \overline A) = P(\Omega) \Rightarrow P(A)+P(\overline A) = 1 \Rightarrow
	P(\overline A)=1-P(A)$.
	\item $\emptyset = \overline \Omega \Rightarrow P(\emptyset) = P(\overline \Omega) = 1-P(\Omega) = 1-1 = 0.$
	\item $B = A\cup (B-A)$. Como $A$ y $B-A$ son incompatibles, $P(B) = P(A\cup (B-A)) = P(A)+P(B-A) \geq
	P(A).$
	
	Si pensamos en probabilidades como áreas, es fácil de ver gráficamente,
	\begin{center}
	\tikzsetnextfilename{probabilidad/inclusion_probabilidad}
	\input{img/probabilidad/probabilidad_inclusion}
	\end{center}
	
	\item $A\subseteq \Omega \Rightarrow P(A)\leq P(\Omega)=1.$
	\item $A=(A-B)\cup (A\cap B)$. Como $A-B$ y $A\cap B$ son incompatibles, $P(A)=P(A-B)+P(A\cap B) \Rightarrow 
	P(A-B)=P(A)-P(A\cap B)$.
	
	Si pensamos en probabilidades como áreas, es fácil de ver gráficamente,
	\begin{center}
	\tikzsetnextfilename{probabilidad/probabilidad_diferencia}
	\input{img/probabilidad/probabilidad_diferencia}
	\end{center}
	
	\item $A\cup B= (A-B) \cup (B-A) \cup (A\cap B)$. As $A-B$, $B-A$ and $A\cap B$ are incompatible, $P(A\cup
	B)=P(A-B)+P(B-A)+P(A\cap B) = P(A)-P(A\cap B)+P(B)-P(A\cap B)+P(A\cap B)= P(A)+P(B)-P(A\cup B)$.
	
	Si pensamos en probabilidades como áreas, es fácil de ver gráficamente,
	\begin{center}
	\tikzsetnextfilename{probabilidad/probabilidad_union}
	\input{img/probabilidad/probabilidad_union}
	\end{center}
	\item $A=\{e_1,\cdots,e_n\} = \{e_1\}\cup \cdots \cup \{e_n\} \Rightarrow P(A)=P(\{e_1\}\cup \cdots \cup \{e_n\}) =
	P(\{e_1\})+ \cdots P(\{e_n\}).$
	\end{enumerate}
	}

\note{
A partir de los axiomas de la definición de probabilidad se pueden deducir las siguientes propiedades.
\begin{enumerate}
\item $P(\overline A) = 1-P(A)$.
\item $P(\emptyset)= 0$.
\item Si $A\subseteq B$ entonces $P(A)\leq P(B)$.
\item $P(A) \leq 1$. Esto sumado al primer axioma restringe el valor de la probabilidad al intervalo real $[0,1]$.
\item Si $A$ y $B$ son sucesos compatibles, es decir, su intersección no es vacía, entonces 
\[P(A\cup B)= P(A) + P(B) - P(A\cap B).\]
Esta es la fórmula que utilizaremos habitualmente para calcular la probabilidad de una unión pues también funciona para sucesos
incompatibles ya que en tal caso la intersección sería el suceso vacío y su probabilidad ya hemos visto que es nula. 
\item Si el suceso $A$ está compuesto por los sucesos elementales $e_1,e_2,...,e_n$, entonces
\[P(A)=\sum_{i=1}^n P(e_i).\]
Esta propiedad es muy interesante ya que si conocemos las probabilidades de todos los elementos del espacio muestral podremos calcular la
probabilidad de cualquier suceso simplemente sumando las probabilidades de los elementos que lo componen. 
\end{enumerate}
}
\end{frame}


%---------------------------------------------------------------------slide----
\begin{frame}
\frametitle{Interpretación de la probabilidad}
Como ha quedado claro en los axiomas anteriores, la probabilidad de un evento $A$ es un número real $P(A)$ que está siempre entre 0 y 1. 

En cierto modo, este número expresa la verosimilitud del evento, es decir, la confianza que hay en que ocurra $A$ en el experimento.
Por tanto, también nos da una medida de la incertidumbre sobre el suceso.
\begin{itemize}
	\item La mayor incertidumbre corresponde a $P(A)=0.5$ (Es tan probable que ocurra $A$ como que no ocurra).
	\item La menor incertidumbre corresponde a $P(A)=1$ ($A$ sucederá con absoluta certeza) y $P(A)=0$ ($A$ no sucederá con absoluta certeza).
\end{itemize} 

Cuando $P(A)$ está más próximo a 0 que a 1, la confianza en que no ocurra $A$ es mayor que la de que ocurra $A$.
Por el contrario, cuando $P(A)$ está más próximo a 1 que a 0, la confianza en que ocurra $A$ es mayor que la de que no ocurra $A$.
\end{frame}


\subsection{Probabilidad condicionada}

%---------------------------------------------------------------------slide----
\begin{frame}
\frametitle{Experimentos condicionados}
En algunas ocasiones, es posible que tengamos alguna información sobre el experimento antes de su realización. 
Habitualmente esa información se da en forma de un suceso $B$ del mismo espacio muestral que sabemos que es cierto antes de realizar el experimento.

En tal caso se dice que el suceso $B$ es un suceso \emph{condicionante}, y la probabilidad de otro suceso $A$ se conoce como \highlight{probabilidad condicionada} y se expresa
\[
	P(A|B).
\]

Esto debe leerse como \emph{probabilidad de $A$ dado $B$} o \emph{probabilidad de $A$ bajo la condición de $B$}.
\end{frame}


%---------------------------------------------------------------------slide----
\begin{frame}
	\frametitle{Experimentos condicionados}
Los condicionantes suelen cambiar el espacio muestral del experimento y por tanto las probabilidades de sus sucesos.

Supongamos que tenemos una muestra de 100 hombres y 100 mujeres con las siguientes frecuencias
\[
\begin{array}{|c|c|c|}
\cline{2-3}
 \multicolumn{1}{c|}{} & \mbox{No fumadores} & \mbox{Fumadores} \\ \hline 
 \rowcolor{color1!30} \mbox{Mujeres} & 80 & 20 \\ \hline
 \mbox{Hombres} & 60 & 40 \\ \hline
\end{array}
\]
Entonces, usando la definición frecuentista de probabilidad, la probabilidad de que una persona elegida al azar sea fumadora es 
\[
	P(\mbox{Fumadora})= \frac{60}{200}=0.3.
\]

\pause 

Sin embargo, si se sabe que la persona elegida es mujer, entonces la muestra se reduce a la primera fila, y la probabilidad de ser fumadora es 
\[
	P(\mbox{Fumadora}|\mbox{Mujer})=\frac{20}{100}=0.2.
\]

\note{
La incertidumbre sobre un suceso depende de la información que se tenga sobre el experimento aleatorio. 
En algunas ocasiones puede que haya que calcular la probabilidad de algún suceso $A$ sabiendo que ha ocurrido otro $B$. En tal caso se dice
que el suceso $B$ es un \emph{condicionante}, y la probabilidad del suceso condicionado suele escribirse como \[P(A|B)\]

Los condicionantes, en el fondo, cambian el espacio muestral del experimento y por tanto las probabilidades de sus sucesos.

\textbf{Ejemplo}. Supongamos que hemos observado las siguientes frecuencias de aprobados en un grupo de 100 hombres y 100 mujeres:
\[
\begin{array}{|c|c|c|}
\cline{2-3}
 \multicolumn{1}{c|}{} & \mbox{Aprobados} & \mbox{Suspensos} \\ \hline 
 \rowcolor{coral} \mbox{Mujeres} & 80 & 20 \\ \hline
 \mbox{Hombres} & 60 & 40 \\ \hline
\end{array}
\]
Entonces, utilizando la definición de frecuentista, la probabilidad de que una persona elegida al azar haya aprobado es
la frecuencia relativa de aprobados que es$P(\mbox{Aprobado})= 140/200=0.7$.

Sin embargo, si se añade información sobre el experimento y nos dicen que la persona elegida es mujer, entonces la muestra se restringiría
sólo a las mujeres y la frecuencia relativa de aprobados en mujeres es $P(\mbox{Aprobado}/\mbox{Mujer})=80/100=0.8$. 
}
\end{frame}


%---------------------------------------------------------------------slide----
\begin{frame}
\frametitle{Probabilidad condicionada}
\begin{definicion}[Probabilidad condicionada]
Dado un espacio muestral $\Omega$ de un experimento aleatorio, y dos dos sucesos $A,B\subseteq \Omega$, la probabilidad de $A$ \emph{condicionada} por $B$ es 
\[ 
	P(A|B) = \frac{P(A\cap B)}{P(B)},
\]
siempre y cuando, $P(B)\neq 0$.
\end{definicion}

Esta definición permite calcular probabilidades sin tener que alterar el espacio muestral original del experimento.

\textbf{Ejemplo}. En el ejemplo anterior 
\[
	P(\mbox{Fumadora}|\mbox{Mujer})= \frac{P(\mbox{Fumadora}\cap \mbox{Mujer})}{P(\mbox{Mujer})} =	\frac{20/200}{100/200}=\frac{20}{100}=0.2.
\]

\note{
El problema de los condicionamientos es que suelen cambiar el espacio muestral de partida. Afortunadamente, es posible calcular
probabilidades condicionadas sin cambiar de espacio muestral gracias a la siguiente fórmula.

\begin{definicion}[Probabilidad condicionada]
Dados dos sucesos $A$ y $B$ de un mismo espacio de sucesos de un experimento aleatorio, la probabilidad de $A$
\emph{condicionada} por $B$ es 
\[ P(A|B) = \frac{P(A\cap B)}{P(B)},\]
siempre y cuando, $P(B)\neq 0$.
\end{definicion}

\textbf{Ejemplo}.  En el ejemplo anterior 
\[
	P(\mbox{Fumadora}|\mbox{Mujer})= \frac{P(\mbox{Fumadora}\cap \mbox{Mujer})}{P(\mbox{Mujer})} =	\frac{20/200}{100/200}=\frac{20}{100}=0.2.
\] 
}
\end{frame}


%---------------------------------------------------------------------slide----
\begin{frame}
\frametitle{Probabilidad del suceso intersección}
A partir de la definición de probabilidad condicionada es posible obtener la fórmula para calcular la probabilidad de la intersección de dos sucesos.
\[
P(A\cap B) = P(A)P(B|A) = P(B)P(A|B).
\]
	
\textbf{Ejemplo}. En una población hay un 30\% de fumadores y se sabe que el 40\% de los fumadores tiene cáncer de pulmón. 
La probabilidad de que una persona elegida al azar sea fumadora y tenga cáncer de pulmón es
\[
	P(\mbox{Fumadora}\cap \mbox{Cáncer})= P(\mbox{Fumadora})P(\mbox{Cáncer}|\mbox{Fumadora}) =
	0.3\times 0.4 = 0.12.
\]
\end{frame}

%---------------------------------------------------------------------slide----
\begin{frame}
\frametitle{Independencia de sucesos}
En ocasiones, la ocurrencia del suceso condicionante no cambia la probabilidad original del suceso principal.
\begin{definicion}[Sucesos independientes]
Dado un espacio muestral $\Omega$ de un experimento aleatorio, dos sucesos $A,B\subseteq \Omega$ son \emph{independientes} si la probabilidad de $A$ no se ve alterada al condicionar por $B$, y viceversa, es decir, 
\[
	P(A|B) = P(A) \quad \mbox{and} \quad P(B|A)=P(B),
\]
si $P(A)\neq 0$ y $P(B)\neq 0$.
\end{definicion}

Esto significa que la ocurrencia de uno evento no aporta información relevante para cambiar la incertidumbre sobre el otro.

Cuando dos eventos son independientes, la probabilidad de su intersección es igual al producto de sus probabilidades,
\[
P(A\cap B) = P(A)P(B).
\]

\note{
En ocasiones, saber que un determinado suceso ha ocurrido no cambia la incertidumbre sobre otro suceso del mismo experimento. Por ejemplo,
si se tiran dos monedas, está claro que el resultado de la primera no cambia la incertidumbre sobre que salga cara en la segunda. En tal
caso se dice que los sucesos son independientes.

Formalmente,
\begin{definicion}[Sucesos independientes]
Dado un espacio muestral $\Omega$ de un experimento aleatorio, dos sucesos $A,B\subseteq \Omega$ son \emph{independientes} si la probabilidad de $A$ no se ve alterada al condicionar por $B$, y viceversa, es decir, 
\[
	P(A|B) = P(A) \quad \mbox{and} \quad P(B|A)=P(B),
\]
si $P(A)\neq 0$ y $P(B)\neq 0$.
\end{definicion}

Si $A$ es independiente de $B$, también se cumple que $B$ es independiente de $A$, y en general simplemente se dice que $A$ y $B$ son
independientes.

También se cumple que si $A$ y $B$ son independientes, entonces 
\[ P(A\cap B) = P(A)P(B|A) = P(A)P(B). \]
}
\end{frame}


\subsection{Espacio probabilístico}

%---------------------------------------------------------------------slide----
\begin{frame}
\frametitle{Espacio probabilístico}
\begin{definicion}[Espacio probabilístico]
Un \emph{espacio probabilístico} de un experimento aleatorio es una terna $(\Omega,\mathcal{F},P)$ donde 
\begin{itemize}
	\item $\Omega$ es el espacio muestral del experimento.
	\item $\mathcal{F}$ es un un conjunto de sucesos del experimento.
	\item $P$ es una función de probabilidad.
\end{itemize} 
\end{definicion}

Si conocemos la probabilidad de todos los elementos de $\Omega$, entonces podemos calcular la probabilidad de cualquier suceso en $\mathcal{F}$ y se puede construir fácilmente el espacio probabilístico.
\end{frame}


%---------------------------------------------------------------------slide----
\begin{frame}
\frametitle{Construcción del espacio probabilístico}
Para determinar la probabilidad de cada suceso elemental se puede utilizar un diagrama de árbol, mediante las siguientes reglas:
\begin{enumerate}
\item Para cada nodo del árbol, etiquetar la rama que conduce hasta él con la probabilidad de que la variable en ese nivel tome el valor del nodo, condicionada por los sucesos correspondientes a sus nodos antecesores en el árbol.
\item La probabilidad de cada suceso elemental en las hojas del árbol es el producto de las probabilidades de las ramas que van desde la raíz a la hoja del árbol.
\end{enumerate}

\begin{center}
	\tikzsetnextfilename{probabilidad/espacio_probabilistico}
	\mode<article>{\resizebox{0.7\textwidth}{!}{\input{img/probabilidad/espacio_probabilistico}}}
	\mode<presentation>{\resizebox{0.9\textwidth}{!}{\input{img/probabilidad/espacio_probabilistico}}}
\end{center}
\end{frame}


% ---------------------------------------------------------------------slide----
\begin{frame}
\frametitle{Árboles de probabilidad con variables dependientes}
\framesubtitle{Ejemplo de dependencia del cáncer con respecto al tabaco}
Sea una población en la que el 30\% de las personas fuman, y que la incidencia del cáncer de pulmón en fumadores es del 40\% mientras que en los no fumadores es del 10\%.

El espacio probabilístico del experimento aleatorio que consiste en elegir una persona al azar y medir las variables Fumar y Cáncer de pulmón se muestra a continuación.

\begin{center}
\tikzsetnextfilename{probabilidad/espacio_probabilistico_fumar_cancer}
\mode<article>{\resizebox{0.6\textwidth}{!}{\input{img/probabilidad/espacio_probabilistico_fumar_cancer}}}
\mode<presentation>{\resizebox{0.8\textwidth}{!}{\input{img/probabilidad/espacio_probabilistico_fumar_cancer}}}
\end{center}

\note{
Sea una población en la que el 30\% de las personas fuman, y que la incidencia del cáncer de pulmón en fumadores es del 40\% mientras que en
los no fumadores es del 10\%.

El árbol de probabilidad que expresa este experimento es el siguiente:

Obsérvese que el fumar o no depende del sexo, así que las ramas que salen del suceso mujer no tienen las mismas probabilidades que las que
salen del suceso hombre. }
\end{frame}


%---------------------------------------------------------------------slide----
\begin{frame}
\frametitle{Árboles de probabilidad con variables independientes}
\framesubtitle{Ejemplo de independencia en el lanzamiento de dos monedas}

El árbol de probabilidad asociado al experimento aleatorio que consiste en el lanzamiento de dos monedas se muestra a continuación.

\begin{center}
	\tikzsetnextfilename{probabilidad/espacio_probabilistico_monedas}
	\mode<article>{\resizebox{0.6\textwidth}{!}{\input{img/probabilidad/espacio_probabilistico_monedas}}}
	\mode<presentation>{\resizebox{0.8\textwidth}{!}{\input{img/probabilidad/espacio_probabilistico_monedas}}}
\end{center}

\note{
El árbol de probabilidad asociado al experimento aleatorio que consiste en el lanzamiento de dos monedas es:

Obsérvese ahora que el resultado de la segunda moneda no depende del resultado de la primera, de manera que las ramas que salen del suceso
cara en la primera moneda tienen las mismas probabilidades que las que salen del suceso cruz.
}
\end{frame}


%---------------------------------------------------------------------slide----
\begin{frame}
\frametitle{Árboles de probabilidad con variables independientes}
\framesubtitle{Ejemplo de independencia en la elección de una muestra aleatoria de tamaño 3}
Dada una población en la que hay un 40\% de hombres y un 60\% de mujeres, el experimento aleatorio que consiste en tomar una muestra
aleatoria de tres personas tiene el árbol de probabilidad que se muestra a continuación.

\begin{center}
	\tikzsetnextfilename{probabilidad/espacio_probabilistico_muestra}
	\mode<article>{\resizebox{0.7\textwidth}{!}{% Author: Alfredo Sánchez Alberca (asalber@ceu.es)

\begin{tikzpicture}[
grow'=right,
% sloped,
level 1/.style ={level distance=2cm, sibling distance=3.2cm, parent anchor=east, child anchor=west},
level 2/.style ={level distance=2cm, sibling distance=1.6cm},
level 3/.style ={level distance=2cm, sibling distance=0.8cm},
level 4/.style ={level distance=1.5cm, sibling distance=0.8cm, dashed},
level 5/.style ={level distance=3cm, sibling distance=0.8cm, dashed},
prob/.style={font=\footnotesize,above}
]

\node (root) {}
	child {node {M}
		child {node {M}
	   		child {node {M}
				child {node{(M,M,M)}
					child {node {$0.216$} edge from parent node[prob] {$0.6\cdot 0.6\cdot 0.6$}} 
				}
				edge from parent node[prob] {$0.6$}
			}
			child {node {H}
				child {node{(M,M,H)}
					child {node {$0.144$} edge from parent node[prob] {$0.6\cdot 0.6\cdot 0.4$}}
				}
				edge from parent node[prob,below] {$0.4$}
			}
			edge from parent node[prob] {$0.6$}
		}
		child {node {H}
	   		child {node {M}
				child {node{(M,H,M)}
					child {node{$0.144$} edge from parent node[prob] {$0.6\cdot 0.4\cdot 0.6$}}
				}
				edge from parent node[prob] {$0.6$}
			}
			child {node {H}
				child {node{(M,H,H)}
					child {node{$0.096$} edge from parent node[prob] {$0.6\cdot 0.4\cdot 0.4$}}
				}
				edge from parent node[prob,below] {$0.4$}
			}
			edge from parent node[prob,below] {$0.4$}
		}
		edge from parent node[prob,left] {$0.6$}
	}
	child {node {H}
		child {node {M}
	   		child {node {M}
				child {node{(H,M,M)}
					child {node {$0.144$} edge from parent node[prob] {$0.4\cdot 0.6\cdot 0.6$}} 
				}
				edge from parent node[prob] {$0.6$}
			}
			child {node {H}
				child {node{(H,M,H)}
					child {node {$0.096$} edge from parent node[prob] {$0.4\cdot 0.6\cdot 0.4$}}
				}
				edge from parent node[prob,below] {$0.4$}
			}
			edge from parent node[prob] {$0.6$}
		}
		child {node {H}
	   		child {node {M}
				child {node{(H,H,M)}
					child {node{$0.096$} edge from parent node[prob] {$0.4\cdot 0.4\cdot 0.6$}}
				}
				edge from parent node[prob] {$0.6$}
			}
			child {node {H}
				child {node{(H,H,H)}
					child {node{$0.064$} edge from parent node[prob] {$0.4\cdot 0.4\cdot 0.4$}}
				}
				edge from parent node[prob,below] {$0.4$}
			}
			edge from parent node[prob,below] {$0.4$}
		}
		edge from parent node[prob,left] {$0.4$}
	};

\begin{scope}[every node/.style={text width=2cm, align=center, anchor=center, font=\bfseries,}]
\node[above= 0.5cm of root-1-1-1-1-1] (labels-level) {Probabilidad};
\node[at =(labels-level-|root-1)] {Persona 1};
\node[at =(labels-level-|root-1-1)] {Persona 2};
\node[at =(labels-level-|root-1-1-1)] {Persona 3};
\node[at =(labels-level-|root-1-1-1-1)]{$\Omega$};
\end{scope}
\end{tikzpicture}
}}
	\mode<presentation>{\resizebox{0.9\textwidth}{!}{% Author: Alfredo Sánchez Alberca (asalber@ceu.es)

\begin{tikzpicture}[
grow'=right,
% sloped,
level 1/.style ={level distance=2cm, sibling distance=3.2cm, parent anchor=east, child anchor=west},
level 2/.style ={level distance=2cm, sibling distance=1.6cm},
level 3/.style ={level distance=2cm, sibling distance=0.8cm},
level 4/.style ={level distance=1.5cm, sibling distance=0.8cm, dashed},
level 5/.style ={level distance=3cm, sibling distance=0.8cm, dashed},
prob/.style={font=\footnotesize,above}
]

\node (root) {}
	child {node {M}
		child {node {M}
	   		child {node {M}
				child {node{(M,M,M)}
					child {node {$0.216$} edge from parent node[prob] {$0.6\cdot 0.6\cdot 0.6$}} 
				}
				edge from parent node[prob] {$0.6$}
			}
			child {node {H}
				child {node{(M,M,H)}
					child {node {$0.144$} edge from parent node[prob] {$0.6\cdot 0.6\cdot 0.4$}}
				}
				edge from parent node[prob,below] {$0.4$}
			}
			edge from parent node[prob] {$0.6$}
		}
		child {node {H}
	   		child {node {M}
				child {node{(M,H,M)}
					child {node{$0.144$} edge from parent node[prob] {$0.6\cdot 0.4\cdot 0.6$}}
				}
				edge from parent node[prob] {$0.6$}
			}
			child {node {H}
				child {node{(M,H,H)}
					child {node{$0.096$} edge from parent node[prob] {$0.6\cdot 0.4\cdot 0.4$}}
				}
				edge from parent node[prob,below] {$0.4$}
			}
			edge from parent node[prob,below] {$0.4$}
		}
		edge from parent node[prob,left] {$0.6$}
	}
	child {node {H}
		child {node {M}
	   		child {node {M}
				child {node{(H,M,M)}
					child {node {$0.144$} edge from parent node[prob] {$0.4\cdot 0.6\cdot 0.6$}} 
				}
				edge from parent node[prob] {$0.6$}
			}
			child {node {H}
				child {node{(H,M,H)}
					child {node {$0.096$} edge from parent node[prob] {$0.4\cdot 0.6\cdot 0.4$}}
				}
				edge from parent node[prob,below] {$0.4$}
			}
			edge from parent node[prob] {$0.6$}
		}
		child {node {H}
	   		child {node {M}
				child {node{(H,H,M)}
					child {node{$0.096$} edge from parent node[prob] {$0.4\cdot 0.4\cdot 0.6$}}
				}
				edge from parent node[prob] {$0.6$}
			}
			child {node {H}
				child {node{(H,H,H)}
					child {node{$0.064$} edge from parent node[prob] {$0.4\cdot 0.4\cdot 0.4$}}
				}
				edge from parent node[prob,below] {$0.4$}
			}
			edge from parent node[prob,below] {$0.4$}
		}
		edge from parent node[prob,left] {$0.4$}
	};

\begin{scope}[every node/.style={text width=2cm, align=center, anchor=center, font=\bfseries,}]
\node[above= 0.5cm of root-1-1-1-1-1] (labels-level) {Probabilidad};
\node[at =(labels-level-|root-1)] {Persona 1};
\node[at =(labels-level-|root-1-1)] {Persona 2};
\node[at =(labels-level-|root-1-1-1)] {Persona 3};
\node[at =(labels-level-|root-1-1-1-1)]{$\Omega$};
\end{scope}
\end{tikzpicture}
}}
\end{center}


\note{
Otro ejemplo de árbol con independencia sería la obtención de una muestra aleatoria con reemplazamiento. 
Dada una población en la que hay un 40\% de hombres y un 60\% de mujeres, el experimento aleatorio que consiste en tomar una muestra
aleatoria con reemplazamiento de tres personas tiene el siguiente árbol de probabilidad:

Obsérvese de nuevo cómo todas las ramas del suceso hombre tienen las mismas probabilidades y lo mismo ocurre con las ramas del suceso mujer. 
}
\end{frame}


\subsection{Teorema de la probabilidad total}

%---------------------------------------------------------------------slide----
\begin{frame}
\frametitle{Sistema completo de sucesos}
\begin{definicion}[Sistema completo de sucesos]
Una colección de sucesos $A_1,A_2,\ldots,A_n$ de un mismo espacio muestral $\Omega$ es un \emph{sistema completo} si cumple las siguientes condiciones:
\begin{enumerate}
\item La unión de todos es el espacio muestral: $A_1\cup \cdots\cup A_n =\Omega$.
\item Son incompatibles dos a dos: $A_i\cap A_j = \emptyset$ $\forall i\neq j$.
\end{enumerate}
\end{definicion}

\begin{center}
	\tikzsetnextfilename{probabilidad/particion_espacio_muestral}
	\input{img/probabilidad/particion_espacio_muestral}
\end{center}

En realidad un sistema completo de sucesos es una partición del espacio muestral de acuerdo a algún atributo, como por ejemplo el sexo o el
grupo sanguíneo.

\note{
En algunos experimentos es posible descomponer el espacio muestral en partes que forman un sistema completo de sucesos. 

\begin{definicion}[Sistema completo de sucesos]
Una colección de sucesos $A_1,A_2,\ldots,A_n$ de un mismo espacio de sucesos es un \emph{sistema completo} si cumple las siguientes condiciones:
\begin{enumerate}
\item La unión de todos es el espacio muestral: $A_1\cup \cdots\cup A_n =E$.
\item Son incompatibles dos a dos: $A_i\cap A_j = \emptyset$ $\forall i\neq j$.
\end{enumerate}
\end{definicion}

En realidad un sistema completo de sucesos es una partición del espacio muestral de acuerdo a algún atributo, como por ejemplo el sexo o el
grupo sanguíneo.
}
\end{frame}


%---------------------------------------------------------------------slide----
\begin{frame}
\frametitle{Teorema de la probabilidad total}
Conocer las probabilidades de un determinado suceso en cada una de las partes de un sistema completo puede ser útil para calcular su probabilidad.
\begin{teorema}[Probabilidad total]
Dado un sistema completo de sucesos $A_1,\ldots,A_n$ y un suceso $B$ de un espacio muestral $\Omega$, la probabilidad de cualquier suceso $B$ del espacio muestral se puede calcular mediante la fórmula
\[
	P(B) = \sum_{i=1}^n P(A_i\cap B) = \sum_{i=1}^n P(A_i)P(B|A_i).
\]
\end{teorema}

\note{
Conocer las probabilidades de un determinado suceso en cada una de las partes de un sistema completo puede ser útil para calcular su
probabilidad.
\begin{teorema}[Probabilidad total]
Dado un sistema completo de sucesos $A_1,\ldots,A_n$ y un suceso $B$ de un mismo espacio de sucesos, se cumple
\[
P(B) = \sum_{i=1}^n P(A_i)P(B/A_i).
\]
\end{teorema}
}
\end{frame}


%---------------------------------------------------------------------slide----
\begin{frame}
\frametitle{Teorema de la probabilidad total}
\framesubtitle{Demostración}
La demostración del teorema es sencilla, ya que al ser $A_1,\ldots,A_n$ un sistema completo tenemos
\[
B = B\cap E = B\cap (A_1\cup \cdots \cup A_n) = (B\cap A_1)\cup \cdots \cup (B\cap A_n)
\]
y como estos sucesos son incompatibles entre sí, se tiene
\begin{align*}
P(B) &= P((B\cap A_1)\cup \cdots \cup (B\cap A_n)) = P(B\cap A_1)+\cdots + P(B\cap A_n) =\\
&= P(A_1)P(B/A_1)+\cdots + P(A_n)P(B/A_n) = \sum_{i=1}^n P(A_i)P(B/A_i).
\end{align*}
	
\begin{center}
	\tikzsetnextfilename{probabilidad/probabilidad_total}
	\mode<article>{\resizebox{0.4\textwidth}{!}{\input{img/probabilidad/probabilidad_total}}}
	\mode<presentation>{\input{img/probabilidad/probabilidad_total}}
\end{center}

\note{
La demostración del teorema es sencilla, ya que al ser $A_1,\ldots,A_n$ un sistema completo tenemos
\[
B = B\cap E = B\cap (A_1\cup \cdots \cup A_n) = (B\cap A_1)\cup \cdots \cup (B\cap A_n)
\]
y como estos sucesos son incompatibles entre sí, se tiene
\begin{align*}
P(B) &= P((B\cap A_1)\cup \cdots \cup (B\cap A_n)) = P(B\cap A_1)+\cdots + P(B\cap A_n) =\\
&= P(A_1)P(B/A_1)+\cdots + P(A_n)P(B/A_n) = \sum_{i=1}^n P(A_i)P(B/A_i).
\end{align*}
}
\end{frame}


%---------------------------------------------------------------------slide----
\begin{frame}
\frametitle{Teorema de la probabilidad total}
\framesubtitle{Un ejemplo de diagnóstico}
Un determinado síntoma $S$ puede ser originado por una enfermedad $E$ pero también lo pueden presentar las personas sin la enfermedad. 
Sabemos que la prevalencia de la enfermedad $E$ es $0.2$. 
Además, se sabe que el $90\%$ de las personas con la enfermedad presentan el síntoma, mientras que sólo el $40\%$ de las personas sin la enfermedad lo presentan.

Si se toma una persona al azar de la población, \emph{¿qué probabilidad hay de que tenga el síntoma?}

Para responder a la pregunta se puede aplicar el teorema de la probabilidad total usando el sistema completo $\{E,\overline{E}\}$:
\[
	P(S) = P(E)P(S|E)+P(\overline E)P(S|\overline E) = 0.2\cdot 0.9 + 0.8\cdot 0.4 = 0.5.
\]
Es decir, la mitad de la población tendrá el síntoma. 

\begin{center}
\emph{¡En el fondo se trata de una media ponderada de probabilidades!} 
\end{center}

\note{
Veamos un ejemplo de aplicación del teorema de la probabilidad total.

Supongamos que un determinado síntoma $B$ puede ser originado por una enfermedad $A$ pero también lo pueden presentar las personas sin
la enfermedad. Sabemos que en la población la tasa de personas con la enfermedad A es $0.2$. Además, de las personas
que presentan la enfermedad, el $90\%$ presentan el síntoma, mientras que de las personas sin la enfermedad sólo lo presentan el $40\%$.

Si se toma una persona al azar de la población, \emph{¿qué probabilidad hay de que tenga el síntoma?}

Para responder a la pregunta hay que fijarse en que el conjunto de sucesos $\{A,\overline{A}\}$ es un sistema completo, ya que $A\cup \overline A = E$ y $A\cap \overline A = \emptyset$, de modo que se puede aplicar el teorema de la probabilidad total:
\[
P(B) = P(A)P(B/A)+P(\overline A)P(B/\overline A) = 0.2\cdot 0.9 + 0.8\cdot 0.4 = 0.5.
\]
Es decir, la mitad de la población tendrá el síntoma. 

\begin{center}
\emph{¡En el fondo se trata de una media ponderada de probabilidades!} 
\end{center}
}
\end{frame}


%---------------------------------------------------------------------slide----
\begin{frame}
\frametitle{Teorema de la probabilidad total}
\framesubtitle{Cálculo con el árbol de probabilidad}
La respuesta a la pregunta anterior es evidente a la luz del árbol de probabilidad del espacio probabilístico del experimento. 

\begin{center}
	\tikzsetnextfilename{probabilidad/espacio_probabilistico_total}
	\mode<article>{\resizebox{0.6\textwidth}{!}{\input{img/probabilidad/espacio_probabilistico_total}}}
	\mode<presentation>{\resizebox{0.8\textwidth}{!}{\input{img/probabilidad/espacio_probabilistico_total}}}
\end{center}

\begin{align*}
	P(S) &= P(E,S) + P(\overline E,S) = P(E)P(S|E)+P(\overline E)P(S|\overline E)\\
		& = 0.2\cdot 0.9+ 0.8\cdot 0.4 = 0.18 + 0.32 = 0.5.
\end{align*}

\note{
El teorema de la probabilidad total también puede deducirse fácilmente a partir del diagrama de árbol de este experimento. 
}
\end{frame}


\subsection{Teorema de Bayes}

%---------------------------------------------------------------------slide----
\begin{frame}
\frametitle{Teorema de Bayes}
Los sucesos de un sistema completo de sucesos $A_1,\cdots,A_n$ también pueden verse como las distintas hipótesis ante un determinado hecho $B$.

En estas condiciones resulta útil poder calcular las probabilidades a posteriori $P(A_i|B)$ de cada una de las hipótesis.

\begin{teorema}[Bayes]
Dado un sistema completo de sucesos $A_1,\ldots,A_n$ y un suceso $B$ de un espacio muestral $\Omega$ y otro suceso $B$ del mismo espacio muestral, la probabilidad de cada suceso $A_i$ $i=1,\ldots,n$ condicionada por $B$ puede calcularse con la siguiente fórmula
\[
	P(A_i|B) = \frac{P(A_i\cap B)}{P(B)} = \frac{P(A_i)P(B|A_i)}{\sum_{i=1}^n P(A_i)P(B|A_i)}.
\]
\end{teorema}

\note{
Los sucesos de un sistema completo de sucesos $A_1,\cdots,A_n$ también pueden verse como las distintas hipótesis ante
un determinado hecho $B$.

En estas condiciones puede ser útil calcular las probabilidades a posteriori $P(A_i/B)$ de cada una de las hipótesis, es decir, una vez
se haya cumplido el suceso $B$. Para ello se utiliza el teorema de Bayes. 

\begin{teorema}[Bayes]
Dado un sistema completo de sucesos $A_1,\ldots,A_n$ y un suceso $B$ de un mismo espacio de sucesos, se cumple
\[
P(A_i/B) = \frac{P(A_i\cap B)}{P(B)} = \frac{P(A_i)P(B/A_i)}{\sum_{i=1}^n P(A_i)P(B/A_i)}.
\]
\end{teorema}
}
\end{frame}


%---------------------------------------------------------------------slide----
\begin{frame}
\frametitle{Teorema de Bayes}
\framesubtitle{Un ejemplo de diagnóstico}
En el ejemplo anterior, una pregunta más interesante es qué diagnosticar a una persona que presenta el síntoma.

En este caso se puede interpretar $E$ y $\overline{E}$ como las dos posibles hipótesis para el síntoma $S$.
Las probabilidades a priori para ellas son $P(E)=0.2$ y $P(\overline E)=0.8$.
Esto quiere decir que si no se dispone de información sobre el síntoma, el diagnóstico será que la persona no tiene la enfermedad.

Sin embargo, si al reconocer a la persona se observa que presenta el síntoma, dicha información condiciona a las hipótesis, y para decidir
entre ellas es necesario calcular sus probabilidades a posteriori, es decir, 
\[
	P(E|S) \mbox{ y } P(\overline{E}|S)
\]

\note{
En el ejemplo anterior se ha visto cómo calcular la probabilidad de que una persona elegida al azar presente el síntoma, pero desde un punto
de vista de diagnóstico clínico, una pregunta más interesante es:
  
Si llega a la consulta una persona que presenta el síntoma, \emph{¿qué se debe diagnosticar?}

En este caso, las hipótesis ante las que hay que decidir son $A$ y $\overline A$ y sus probabilidades ``a priori'' son $P(A)=0.2$ y
$P(\overline A)=0.8$.

Esto quiere decir que si no hubiese ninguna información sobre la persona, el diagnóstico sería que no tiene la enfermedad pues es mucho más
probable que que la tenga.

Sin embargo, si al reconocer a la persona se observa que presenta el síntoma, dicha información condiciona a las hipótesis, y para decidir
entre ellas es necesario calcular sus probabilidades ``a posteriori'', es decir 
\[ P(A|B) \mbox{ y } P(\overline A|B)\]
}
\end{frame}


%---------------------------------------------------------------------slide----
\begin{frame}
\frametitle{Teorema de Bayes}
\framesubtitle{Un ejemplo de diagnóstico}
Para calcular las probabilidades a posteriori se puede utilizar el teorema de Bayes:
\begin{align*}
P(E|S) &= \frac{P(E)P(S|E)}{P(E)P(S|E)+P(\overline{E})P(S|\overline{E})} = \frac{0.2\cdot 0.9}{0.2\cdot 0.9 + 0.8\cdot 0.4} = \frac{0.18}{0.5}=0.36,\\
P(\overline{E}|S) &= \frac{P(\overline{E})P(S|\overline{E})}{P(E)P(S|E)+P(\overline{E})P(S|\overline{E})} = \frac{0.8\cdot 0.4}{0.2\cdot 0.9 + 0.8\cdot 0.4} = \frac{0.32}{0.5}=0.64.
\end{align*}

Como se puede ver la probabilidad de tener la enfermedad ha aumentado.
No obstante, la probabilidad de no tener la enfermedad sigue siendo mayor que la de tenerla, y por esta razón el diagnóstico seguirá siendo que no tiene la enfermedad.

En este caso se dice que el síntoma $S$ \emph{no es determinante} a la hora de diagnosticar la enfermedad.

\note{
Para calcular las probabilidades ``a posteriori'' se puede utilizar el teorema de Bayes:
\begin{align*}
P(A|B) &= \frac{P(A)P(B/A)}{P(A)P(B/A)+P(\overline A)P(B/\overline A)} = \frac{0.2\cdot 0.9}{0.2\cdot 0.9 + 0.8\cdot 0.4} = \frac{0.18}{0.5}=0.36,\\
P(\overline A|B) &= \frac{P(\overline A)P(B/\overline A)}{P(A)P(B/A)+P(\overline A)P(B/\overline A)} = \frac{0.8\cdot 0.4}{0.2\cdot 0.9 + 0.8\cdot 0.4} = \frac{0.32}{0.5}=0.64.
\end{align*}
Según esto, a pesar de que la probabilidad de estar enfermo ha aumentado, seguiríamos diagnosticando que no lo está, puesto que es más
probable.

En este caso se dice que el síntoma $B$ \emph{no es determinante} a la hora de diagnosticar la enfermedad, pues la información que aporta no
sirve para cambiar el diagnóstico en ningún caso.
}
\end{frame}


\subsection{Epidemiología}

%---------------------------------------------------------------------slide----
\begin{frame}
\frametitle{Epidemiología}
Una de las ramas de la Medicina que hace un mayor uso de la probabilidad es la \highlight{Epidemiología}, que estudia la distribución y las causas de las enfermedades en las poblaciones, identificando factores de riesgos para las enfermedades de cara a la atención médica preventiva. 

En Epidemiología interesa la frecuencia de un \emph{suceso médico} $E$ (típicamente una enfermedad como la gripe, un factor de riesgo como fumar o un factor de protección como vacunarse) que se mide mediante una variable nominal con dos categorías (ocurrencia o no del suceso).

Hay diferentes medidas relativas a la frecuencia de un suceso médico.
Las más importantes son:

\begin{itemize}
  \item Prevalencia
  \item Incidencia
  \item Riesgo relativo
  \item Odds ratio
\end{itemize}
\end{frame}


%---------------------------------------------------------------------slide----
\begin{frame}
\frametitle{Prevalencia}
\begin{definicion}[Prevalencia]
La \emph{prevalencia} de un suceso médico $E$ es la proporción de una población que está afectada por el suceso. 
\[
  \mbox{Prevalencia}(E) = \frac{\mbox{Nº individuos afectados por $E$}}{\mbox{Tamaño poblacional}}
\]
\end{definicion}

A menudo, la prevalencia se estima mediante una muestra como la frecuencia relativa de los individuos afectados por el suceso en la muestra.
Es también común expresarla est frecuencia como un porcentaje. 

\textbf{Ejemplo}. Para estimar la prevalencia de la gripe se estudió una muestra de 1000 personas de las que 150 presentaron gripe.
Así, la prevalencia de la gripe es aproximadamente 150/1000=0.15, es decir, un 15\%. 
\end{frame}
  

%---------------------------------------------------------------------slide----
\begin{frame}
\frametitle{Incidencia}
La \highlight{incidencia} mide la probabilidad de ocurrencia de un suceso médico en una población durante un periodo de tiempo específico.
La incidencia puede medirse como una proporción acumulada o como una tasa.

\begin{definicion}[Incidencia acumulada]
La \emph{incidencia acumulada} de un suceso médico $E$ es la proporción de individuos que experimentaron el evento en un periodo de tiempo, es decir, el número de nuevos casos afectados por el evento en el periodo de tiempo, divido por el tamaño de la población inicialmente en riesgo de verse afectada. 
\[
  R(E)=\frac{\mbox{Nº de nuevos casos con $E$}}{\mbox{Tamaño de la población en riesgo}}
\]
\end{definicion}

\textbf{Ejemplo}. Una población contenía inicialmente $1000$ personas sin gripe y después de dos años se observó que 160 de ellas sufrieron gripe.
La incidencia acumulada de la gripe es 160 casos pro 1000 personas por dos años, es decir, 16\% en dos años.
\end{frame}
  

%---------------------------------------------------------------------slide----
\begin{frame}
\frametitle{Tasa de indicencia o Riesgo absoluto}
  
\begin{definicion}[Tasa de incidencia]
La \emph{tasa de incidencia} o \emph{riesgo absoluto} de un suceso médico $E$ es el número de nuevos casos afectados por el evento divido por la población en riesgo y por el número de unidades temporales del periodo considerado.
\[
  R(E)=\frac{\mbox{Nº nuevos casos con $E$}}{\mbox{Tamaño población en riesgo}\times \mbox{Nº unidades de tiempo}}
\]
\end{definicion}

\textbf{Ejemplo}. Una población contenía inicialmente $1000$ personas sin gripe y después de dos años se observó que 160 de ellas sufrieron gripe.
Si se considera el año como intervalo de tiempo, la tasa de incidencia de la gripe es 160 casos dividida por 1000 personas y por dos años, es decir, 80 casos por 1000 personas-año o 8\% de personas al año.
\end{frame}


%---------------------------------------------------------------------slide----
\begin{frame}
\frametitle{Prevalencia vs Incidencia}
La prevalencia no debe confundirse con la incidencia. 
La prevalencia indica cómo de extendido está el suceso médico en una población, sin preocuparse por cuándo los sujetos se han expuesto al riesgo o durante cuánto tiempo, mientras que la incidencia se fija en el riesgo de verse afectado por el suceso en un periodo concreto de tiempo.

Así, la prevalencia se calcula en estudios transversales en un momento temporal puntual, mientras que para medir la incidencia se necesita un estudio longitudinal que permita observar a los individuos durante un periodo de tiempo.

La incidencia es más útil cuando se pretende entender la causalidad del suceso: por ejemplo, si la incidencia de una enfermedad en una población aumenta, seguramente hay un factor de riesgo que lo está promoviendo.

Cuando la tasa de incidencia es aproximadamente constante en la duración del suceso, la prevalencia es aproximadamente el producto de la incidencia por la duración media del suceso, es decir,
\begin{center}
  prevalencia = incidencia $\times$ duración
\end{center}
\end{frame}
  

%---------------------------------------------------------------------slide----
\begin{frame}
\frametitle{Comparación de riesgos}
Para determinar si un factor o característica está asociada con el suceso médico es necesario comparar el riesgo del suceso en dos poblaciones, una expuesta a el factor y la otra no. 
El grupo expuesto a el factor se conoce como \emph{grupo tratamiento} o \emph{grupo experimental} y el grupo no expuesto como \emph{grupo control}.

Habitualmente los casos observados para cada grupo se representan en una tabla de 2$\times$2 como la siguiente:

\begin{center}
  \begin{tabular}{|m{3cm}|m{3cm}<{\centering}|m{3cm}<{\centering}|}
  \cline{2-3}
  \multicolumn{1}{c|}{} & Suceso $E$ & No suceso $\overline E$\\ 
  \hline
  Grupo tratamiento\newline (expuestos) & $a$ & $b$\\ 
  \hline 
  Grupo control\newline (no expuestos) & $c$ & $d$\\ 
  \hline
  \end{tabular}
\end{center}
\end{frame}
 

%---------------------------------------------------------------------slide----
\begin{frame}
\frametitle{Riesgo atribuible o diferencia de riesgos $RA$}
\begin{definicion}[Riesgo atribuible]
El \emph{riesgo atribuible} o \emph{diferencia de riesgo} de un suceso médico $E$ para los individuos expuestos a un factor es la diferencia entre los riesgos absolutos de los grupos tratamiento y control.
\[
	RA(E)=R_T(E)-R_C(E)=\frac{a}{a+b}-\frac{c}{c+d}.
\]
\end{definicion}

El riesgo atribuible es el riesgo de un suceso que es debido específicamente al factor de interés.

Obsérvese que el riesgo atribuible puede ser positivo, cuando el riesgo del grupo tratamiento es mayor que el del grupo control, o negativo, de lo contrario.
\end{frame}


%---------------------------------------------------------------------slide----
\begin{frame}
\frametitle{Riesgo relativo $RR$}
\framesubtitle{Ejemplo de una vacuna}
Para determinar la efectividad de una vacuna contra la gripe, una muestra de 1000 personas sin gripe fueron seleccionadas al comienzo del año. 
La mitad de ellas fueron vacunadas (grupo tratamiento) y la otra mitad recibieron un placebo (grupo control).
La tabla siguiente resume los resultados al final del año.

\begin{center}
	\begin{tabular}{|m{3cm}|m{1.8cm}<{\centering}|m{1.8cm}<{\centering}|}
	\cline{2-3}
	\multicolumn{1}{c|}{} & Gripe $E$ & No gripe $\overline E$\\ 
	\hline
	Grupo tratamiento\newline (vacunados) & $20$ & $480$\\ 
	\hline 
	Grupo control\newline (No vacunados) & $80$ & $420$\\ 
	\hline
\end{tabular}
\end{center}

El riesgo atribuible de contraer la gripe cuando se es vacunado es 
\[
	AR(D) = \frac{20}{20+480}-\frac{80}{80+420} = -0.12.
\]
Esto quiere decir que el riesgo de contraer la gripe es un 12\% menor en vacunados que en no vacunados.
\end{frame}

	
%---------------------------------------------------------------------slide----
\begin{frame}
\frametitle{Riesgo relativo $RR$}
\begin{definicion}[Riesgo relativo]
El \emph{riesgo relativo} de un suceso médico $E$ para los individuos expuestos a un factor es el cociente entre las proporciones de individuos afectados por el suceso en un periodo de tiempo de los grupos tratamiento y control.
Es decir, el cociente entre las incidencias de grupo tratamiento y el grupo control.
\[
  RR(D)=\frac{\mbox{Riesgo grupo tratamiento}}{\mbox{Riesgo grupo control}}=\frac{R_T(E)}{R_C(E)}=\frac{a/(a+b)}{c/(c+d)}
\]
\end{definicion}

El riesgo relativo compara el riesgo de desarrollar un suceso médico entre el grupo tratamiento y el grupo control.
\begin{itemize}
  \item $RR=1$ $\Rightarrow$ No hay asociación entre el suceso y la exposición al factor.
  \item $RR<1$ $\Rightarrow$ La exposición al factor disminuye el riesgo del suceso.
  \item $RR>1$ $\Rightarrow$ La exposición al factor aumenta el riesgo del suceso.
\end{itemize}
Cuanto más lejos de 1, más fuerte es la asociación. 
\end{frame}


%---------------------------------------------------------------------slide----
\begin{frame}
\frametitle{Riesgo relativo $RR$}
\framesubtitle{Ejemplo de una vacuna}
Para determinar la efectividad de una vacuna contra la gripe, una muestra de 1000 personas sin gripe fueron seleccionadas al comienzo del año. 
La mitad de ellas fueron vacunadas (grupo tratamiento) y la otra mitad recibieron un placebo (grupo control).
La tabla siguiente resume los resultados al final del año.

\begin{center}
  \begin{tabular}{|m{3cm}|m{1.8cm}<{\centering}|m{1.8cm}<{\centering}|}
  \cline{2-3}
  \multicolumn{1}{c|}{} & Gripe $E$ & No gripe $\overline E$\\ 
  \hline
  Grupo tratamiento\newline (vacunados) & $20$ & $480$\\ 
  \hline 
  Grupo control\newline (No vacunados) & $80$ & $420$\\ 
  \hline
\end{tabular}
\end{center}

El riesgo relativo de contraer la gripe cuando se es vacunado es 
\[
  RR(D) = \frac{20/(20+480)}{80/(80+420)} = 0.25.
\]
Así, la probabilidad de contraer la gripe en los individuos vacunados fue la cuarta parte de la de contraerla en el caso de no haberse vacunado, es decir, la vacuna reduce el riesgo de gripe un 75\%.
\end{frame}


%---------------------------------------------------------------------slide----
\begin{frame}
\frametitle{Odds}
Una forma alternativa de medir el riesgo de un suceso médico es el \emph{odds}.

\begin{definicion}[Odie]
El \emph{odds} de un suceso médico $E$ en una población es el cociente entre el número de individuos que adquirieron el suceso y los que no en un periodo de tiempo.
\[
	ODDS(E)=\frac{\mbox{Nº nuevos casos con $E$}}{\mbox{Nº casos sin $E$}}=\frac{P(E)}{P(\overline E)}
\]
\end{definicion}

A diferencia de la incidencia, que es una proporción menor o igual que 1, el odds puede ser mayor que 1.
No obstante es posible convertir el odds en una probabilidad con al fórmula
\[
  P(E) = \frac{ODDS(E)}{ODDS(E)+1}
\]

\textbf{Ejemplo} Una población contenía inicialmente $1000$ personas sin gripe.
Después de un año 160 de ellas tuvieron gripe.
Entonces el odds de la gripe es 160/840.

Obsérvese que la incidencia es 160/1000.
\end{frame}
 

%---------------------------------------------------------------------slide----
\begin{frame}
\frametitle{Odds ratio $OR$ }
\begin{definicion}[Odds ratio]
El \emph{odds ratio} o la \emph{oportunidad relativa} de un suceso médico $E$ para los individuos expuestos a un factor es el cociente entre los odds del sucesos de los grupos tratamiento y control.
\[
  OR(E)=\frac{\mbox{Odds en grupo tratamiento}}{\mbox{Odds en grupo control}}=\frac{a/b}{c/d}=\frac{ad}{bc}
\]
\end{definicion}

El odds ratio compara los odds de un suceso médico entre el grupo tratamiento y control.
La interpretación es similar a la del riesgo relativo:
\begin{itemize}
  \item $OR=1$ $\Rightarrow$ No existe asociación entre el suceso y la exposición al factor.
  \item $OR<1$ $\Rightarrow$ La exposición al factor disminuye el riesgo del suceso.
  \item $OR>1$ $\Rightarrow$ La exposición al factor aumenta el riesgo del suceso.
\end{itemize}
Cuanto más lejos de 1, más fuerte es la asociación. 
\end{frame}


%---------------------------------------------------------------------slide----
\begin{frame}
\frametitle{Odds ratio $OR$}
\framesubtitle{Ejemplo de una vacuna}
Para determinar la efectividad de una vacuna contra la gripe, una muestra de 1000 personas sin gripe fueron seleccionadas al comienzo del año. 
La mitad de ellas fueron vacunadas (grupo tratamiento) y la otra mitad recibieron un placebo (grupo control).
La tabla siguiente resume los resultados al final del año.

\begin{center}
  \begin{tabular}{|m{3cm}|m{1.8cm}<{\centering}|m{1.8cm}<{\centering}|}
  \cline{2-3}
  \multicolumn{1}{c|}{} & Gripe $D$ & No gripe $\overline D$\\ 
  \hline
  Grupo tratamiento\newline (vacunados) & $20$ & $480$\\ 
  \hline 
  Grupo control\newline (No vacunados) & $80$ & $420$\\ 
  \hline
\end{tabular}
\end{center}

El odds ratio de sufrir la gripe para los individuos vacunados es
\[
  OR(D) = \frac{20/480}{80/420} = 0.21875.
\]
Esto quiere decir que el odds de sufrir la gripe frente a no sufrirla en los vacunados es casi un quinto del de los no vacunados, es decir, que aproximadamente por cada 22 personas vacunadas con gripe habrá 100 personas no vacunadas con gripe.
\end{frame}


\begin{frame}
\frametitle{Riesgo relativo vs Odds ratio} 
El riesgo relativo y el odds ratio son dos medidas de asociación pero su interpretación es ligeramente diferente. 
Mientras que el riesgo relativo expresa una comparación de riesgos entre los grupos tratamiento y control, el odds ratio expresa una comparación de odds, que no es lo mismo que el riesgo.
Así, un odds ratio de 2 \emph{no} significa que el grupo tratamiento tiene el doble de riesgo de adquirir el suceso.

La interpretación del odds ratio es un poco más enrevesada porque es contrafactual, y nos da cuántas veces es más frecuente el suceso en el grupo tratamiento en comparación con el control, asumiendo que en el grupo control es tan frecuente que ocurra el suceso como que no.  

La ventaja del odds ratio es que no depende de la prevalencia o la incidencia del suceso, y debe usarse siempre que el número de individuos que presenta el suceso se selecciona arbitrariamente en ambos grupos, como ocurre en los estudios casos-control.
\end{frame}


\begin{frame}
\frametitle{Riesgo relativo vs Odds ratio}
\framesubtitle{Ejemplo de cáncer de pulmón y fumar}
Para determinar la asociación entre el cáncer de pulmón y fumar se tomaron dos muestras (la segunda con el doble de individuos sin cáncer) obteniendo los siguientes resultados:

\bigskip
\begin{columns}
\begin{column}{0.5\textwidth}
\textbf{Sample 1}
\begin{center}
\small
\begin{tabular}{|m{1.9cm}|m{1.2cm}<{\centering}|m{1.5cm}<{\centering}|}
\cline{2-3}
\multicolumn{1}{c|}{} & Cáncer & No cáncer\\ 
\hline
Fumadores & $60$ & $80$\\ 
\hline 
No fumadores & $40$ & $320$\\ 
\hline
\end{tabular}
\end{center}

\begin{align*}
RR(D) &= \frac{60/(60+80)}{40/(40+320)} = 3.86.\\
OR(D) &= \frac{60/80}{40/320} = 6. 
\end{align*}
\end{column}
\begin{column}{0.5\textwidth}
\textbf{Sample 2}
\begin{center}
\small
\begin{tabular}{|m{1.9cm}|m{1.2cm}<{\centering}|m{1.5cm}<{\centering}|}
\cline{2-3}
\multicolumn{1}{c|}{} & Cáncer & No cáncer\\ 
\hline
Fumadores & $60$ & $160$\\ 
\hline 
No fumadores & $40$ & $640$\\ 
\hline
\end{tabular}
\end{center}

\begin{align*}
RR(D) &= \frac{60/(60+160)}{40/(40+640)} = 4.64.\\
OR(D) &= \frac{60/160}{40/640} = 6. 
\end{align*}
\end{column}
\end{columns}
Así, cuando cambia la incidencia o prevalencia de un suceso (cáncer de pulmón) el riesgo relativo cambia, mientras que el odds ratio no.
\end{frame}


\begin{frame}
\frametitle{Riesgo relativo vs Odds ratio} 
La relación entre el riesgo relativo y el odds ratio viene dada por la siguiente fórmula 

\[
  RR = \frac{OR}{1-R_0+R_0*OR}=OR*\frac{1-R_1}{1-R_0},
\]

donde $R_C$ and $R_T$ son la prevalencia o la incidencia en los grupos control y tratamiento respectivamente.

El odds ratio siempre sobrestima el riesgo relativo cuando este es mayor que 1 y lo subestima cuando es menor que 1.
No obstante, con sucesos médicos raros (con una prevalencia o incidencia baja) el riesgo relativo y el odds ratio son casi iguales.
\end{frame}


\begin{frame}
\frametitle{Riesgo relativo vs Odds ratio} 
\begin{center}
\tikzsetnextfilename{probabilidad/odds_ratio_vs_riesgo_relativo}
\mode<article>{\resizebox{0.6\textwidth}{!}{\input{img/probabilidad/odds_ratio_vs_riesgo_relativo}}}
\mode<presentation>{\resizebox{!}{0.95\textheight}{\input{img/probabilidad/odds_ratio_vs_riesgo_relativo}}}
\end{center}
\end{frame}


\subsection{Tests diagnósticos}

%---------------------------------------------------------------------slide----
\begin{frame}
\frametitle{Tests diagnósticos}
En Epidemiología es común el uso de test para diagnosticar enfermedades.

Generalmente estos test no son totalmente fiables, sino que hay cierta probabilidad de acierto o fallo en el diagnóstico, que suele
representarse en la siguiente tabla:
\begin{center}
	\begin{tabular}{|m{2.5cm}|m{3cm}<{\centering}|m{3.1cm}<{\centering}|}
		\cline{2-3}
		\multicolumn{1}{c|}{} & Presencia enfermedad $E$ & Ausencia enfermedad $\overline D$\\ \hline
		Test positivo $+$ & \textcolor{green}{Verdadero positivo}\newline $VP$ & \textcolor{red}{Falso positivo}\newline $FP$\\ \hline Test negativo $-$ & \textcolor{red}{Falso negativo}\newline $FN$ &
		\textcolor{green}{Verdadero Negativo}\newline $VN$\\ \hline
	\end{tabular}
\end{center}

\note{
En epidemiología es común el uso de test para diagnosticar enfermedades.

Generalmente estos test no son totalmente fiables, sino que hay cierta probabilidad de acierto o fallo en el diagnóstico, que suele
representarse en la siguiente tabla:
\begin{center}
	\begin{tabular}{|m{2.5cm}|m{3cm}<{\centering}|m{3.1cm}<{\centering}|}
		\cline{2-3}
		\multicolumn{1}{c|}{} & Presencia enfermedad $E$ & Ausencia enfermedad $\overline D$\\ \hline
		Test positivo $+$ & \textcolor{green}{Verdadero positivo}\newline $VP$ & \textcolor{red}{Falso positivo}\newline $FP$\\ \hline Test negativo $-$ & \textcolor{red}{Falso negativo}\newline $FN$ &
		\textcolor{green}{Verdadero negativo}\newline $VN$\\ \hline
	\end{tabular}
\end{center}
}
\end{frame}


%---------------------------------------------------------------------slide----
\begin{frame}
	\frametitle{Sensibilidad y especificidad de un test diagnóstico}
	La fiabilidad de un test diagnóstico depende de las siguientes probabilidades.
	\begin{definicion}[Sensibilidad]
	La \emph{sensibilidad} de un test diagnóstico es la proporción de resultados positivos del test en personas con la enfermedad,
	\[
	P(+|E)=\frac{VP}{VP+FN}
	\]
	\end{definicion}
	
	\begin{definicion}[Especificidad]
	La \emph{especificidad} de un test diagnóstico es la proporción de resultados negativos del test en personas sin la enfermedad,
	\[
	P(-|\overline{E})=\frac{VN}{VN+FP}
	\]
	\end{definicion}
	\end{frame}


%---------------------------------------------------------------------slide----
\begin{frame}
\frametitle{Interpretación de la sensibilidad y la especificidad}
Normalmente existe un balance entre la sensibilidad y la especificidad. 
	
Un test con una alta sensibilidad detectará la enfermedad en la mayoría de las personas enfermas, pero también dará más falsos positivos que un test menos sensible.
De este modo, un resultado positivo en un test con una gran sensibilidad no es muy útil para confirmar la enfermedad, pero un resultado negativo es útil para descartar la enfermedad, ya que raramente da resultados negativos en personas con la enfermedad.

Por otro lado, un test con una alta especificidad descartará la enfermedad en la mayoría de las personas sin la enfermedad, pero también producirá más falsos negativos que un test menos específico.
Así, un resultado negativo en un test con una gran especificidad no es útil para descartar la enfermedad, pero un resultado positivo es muy útil para confirmar la enfermedad, ya que raramente da resultados positivos en personas sin la enfermedad.
\end{frame}


%---------------------------------------------------------------------slide----
\begin{frame}
\frametitle{Interpretación de la sensibilidad y la especificidad}
Decidir entre un test con una gran sensibilidad o un test con una gran especificidad depende del tipo de enfermedad y el objetivo del test.
En general, utilizaremos un test sensible cuando:
	\begin{itemize}
	\item La enfermedad es grave y es importante detectarla.
	\item La enfermedad es curable.
	\item Los falsos positivos no provocan traumas serios.
\end{itemize}
	
Y utilizaremos un test específico cuando:
\begin{itemize}
	\item La enfermedad es importante pero difícil o imposible de curar.
	\item Los falsos positivos pueden provocar traumas serios.
	\item El tratamiento de los falsos positivos puede tener graves consecuencias. 
	\end{itemize}
	\end{frame}
	

%---------------------------------------------------------------------slide----
\begin{frame}
\frametitle{Valores predictivos de un test diagnóstico}
Pero el aspecto más importante de un test diagnóstico es su poder predictivo, que se mide con las siguientes probabilidades a posteriori.

\begin{definicion}[Valor predictivo positivo $VPP$]
El \emph{valor predictivo positivo} de un test diagnóstico es la proporción de personas con la enfermedad entre las personas con resultado positivo en el test,
\[
	P(E|+) = \frac{VP}{VP+FP}
\]
\end{definicion}
	
\begin{definicion}[Valor predictivo negativo $VPN$]
El \emph{valor predictivo negativo} de un test diagnóstico es la proporción de personas sin la enfermedad entre las personas con resultado negativo en el test,
\[
	P(\overline{E}|-) = \frac{VN}{VN+FN}
\]
\end{definicion}
\end{frame}


%---------------------------------------------------------------------slide----
\begin{frame}
\frametitle{Interpretación de los valores predictivos}
Los valores predictivos positivo y negativo permiten confirmar o descartar la enfermedad, respectivamente, si alcanzan al menos el umbral de $0.5$.
\[
	\begin{array}{rcl}
	VPP>0.5 & \Rightarrow & \mbox{Diagnosticar la enfermedad}\\
	VPN>0.5 & \Rightarrow & \mbox{Diagnosticar la no enfermedad} 
	\end{array}
\]
	
No obstante, estas probabilidades dependen de la proporción de personas con la enfermedad en la población $P(E)$ que se conoce como \highlight{prevalencia} de la enfermedad.
Pueden calcularse a partir de la sensibilidad y la especificidad del test diagnóstico usando el teorema de Bayes.
	
\begin{align*}
VPP=P(E|+) &= \frac{P(E)P(+|E)}{P(E)P(+|E)+P(\overline{E})P(+|\overline{E})}\\
VPN=P(\overline{E}|-) &= \frac{P(\overline{E})P(-|\overline{E})}{P(E)P(-|E)+P(\overline{E})P(-|\overline{E})}
%  = \frac{\mbox{Prevalence}\cdot
% \mbox{Sensitivity}}{\mbox{Prevalence}\cdot \mbox{Sensitivity}+(1-\mbox{Prevalence})\cdot (1-\mbox{Specificity})}
\end{align*} 
	
Así, con enfermedades frecuentes, el valor predictivo positivo aumenta, y con enfermedades raras, el valor predictivo negativo aumenta.
\end{frame}


% ---------------------------------------------------------------------slide----
\begin{frame}
\frametitle{Test diagnósticos}
\framesubtitle{Ejemplo}
Un test diagnóstico para la gripe se ha aplicado a una muestra aleatoria de 1000 personas.
Los resultados aparecen resumidos en la siguiente tabla.
\begin{center}
	\begin{tabular}{|m{2.5cm}|m{3.2cm}<{\centering}|m{3.2cm}<{\centering}|}
	\cline{2-3}
	\multicolumn{1}{c|}{} & Presencia de gripe $E$ & Ausencia de gripe $\overline E$\\ \hline
	Test $+$ & 95 & 90 \\
	\hline
	Test $-$ & 5 & 810 \\
	\hline
	\end{tabular}
\end{center}
	
Según esta muestra, la prevalencia de la gripe puede estimarse como
\[
	P(E) = \frac{95+5}{1000} = 0.1.
\] 
	
La sensibilidad del test diagnóstico es 
\[
	P(+|E) = \frac{95}{95+5}= 0.95. 
\] 
	
Y la especificidad es 
\[
	P(-|\overline{E}) = \frac{810}{90+810}=0.9.
\]
\end{frame}
	
	
% ---------------------------------------------------------------------slide----
\begin{frame}
\frametitle{Test diagnósticos}
\framesubtitle{Continuación del ejemplo}
El valor predictivo positivo del test es 
\[
	VPP = P(E|+) = \frac{95}{95+90} = 0.5135.
\]
	
Como este valor es mayor que $0.5$, eso significa que se diagnosticará la gripe si el resultado del test es positivo.
No obstante, la confianza en el diagnóstico será baja, ya que el valor es poco mayor que $0.5$.
	
Por otro lado, el valor predictivo negativo es 
\[
	VPN = P(\overline{E}|-) = \frac{810}{5+810} = 0.9939. 
\]
	
Como este valor es casi 1, eso significa que es casi seguro que no se tiene la gripe cuando el resultado del test es negativo.
	
Así, se puede concluir que este test es muy potente para descartar la gripe, pero no lo est tanto para confirmarla.
\end{frame}
	
	
%---------------------------------------------------------------------slide----
\begin{frame}
\frametitle{Razón de verosimilitud de un test diagnóstico}
La siguientes medidas también se derivan de la sensibilidad y la especificidad de un test diagnóstico.
\begin{definicion}[Razón de verosimilitud positiva $RV+$]
La \emph{razón de verosimilitud positiva} de un test diagnóstico es el cociente entre la probabilidad de un resultado positivo en personas con la enfermedad y personas sin la enfermedad, respectivamente.
\[
	RV+=\frac{P(+|E)}{P(+|\overline{E})} = \frac{\mbox{Sensibilidad}}{1-\mbox{Especificidad}}
\]
\end{definicion}
	
\begin{definicion}[Razón de verosimilitud negativa $RV-$]
La \emph{razón de verosimilitud negativa} de un test diagnóstico es el cociente entre la probabilidad de un resultado negativo en personas con la enfermedad y personas sin la enfermedad, respectivamente.
\[
	RV-=\frac{P(-|E)}{P(-|\overline{E})} = \frac{1-\mbox{Sensibilidad}}{\mbox{Especificidad}}
\]
\end{definicion}
\end{frame}
	
	
%---------------------------------------------------------------------slide----
\begin{frame}
\frametitle{Interpretación de las razones de verosimilitud}
La razón de verosimilitud positiva puede interpretarse como el número de veces que un resultado positivo es más probable en personas con la enfermedad que en personas sin la enfermedad.
	
Por otro lado, la razón de verosimilitud negativa puede interpretarse como el número de veces que un resultado negativo es más probable en personas con la enfermedad que en personas sin la enfermedad.
	
Las probabilidades a posteriori pueden calculares a partir de las probabilidades a priori usando las razones de verosimilitud
	
\[
	P(E|+) = \frac{P(E)P(+|E)}{P(E)P(+|E)+P(\overline{E})P(+|\overline{E})} = \frac{P(E)RV+}{1-P(E)+P(E)RV+}
\]
	
Así,
\begin{itemize}
	\item Una razón de verosimilitud positiva mayor que 1 aumenta la probabilidad de la enfermedad.
	\item Una razón de verosimilitud positiva menor que 1 disminuye la probabilidad de la enfermedad.
	\item Una razón de verosimilitud 1 no cambia la probabilidad a priori de la de tener la enfermedad.
\end{itemize}
\end{frame}
	
	
%---------------------------------------------------------------------slide----
\begin{frame}
\frametitle{Interpretación de las razones de verosimilitud}
\begin{center}
	\tikzsetnextfilename{probabilidad/razon_verosimilitud}
	\mode<article>{\resizebox{0.6\textwidth}{!}{\input{img/probabilidad/razon_verosimilitud}}}
	\mode<presentation>{\resizebox{0.75\textwidth}{!}{\input{img/probabilidad/razon_verosimilitud}}}
\end{center}
\end{frame}