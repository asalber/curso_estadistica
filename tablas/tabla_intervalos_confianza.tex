\documentclass[a4paper,landscape]{article}
%%%%%%%%%%%%%%%%%%%%%%%%%%%%
\usepackage[spanish]{babel}
\usepackage[utf8x]{inputenc}
\usepackage{amsmath}
\usepackage{macros}
\usepackage{graphicx}
\usepackage[top=2.3cm, bottom=2.3cm, left=2.54cm, right=2.54cm]{geometry}
\usepackage{fancyhdr}
\pagestyle{fancy}


\lhead{\textsc{\textcolor[rgb]{0.00,0.00,0.50}{Universidad San Pablo CEU}}}
\rhead{\textsl{\textsf{\textcolor[rgb]{0.00,0.00,0.50}{Departamento de Métodos Cuantitativos}}}}
\renewcommand{\headrulewidth}{0pt}

\begin{document}

\begin{center}
\section*{Intervalos de Confianza}
\begin{tabular}{|c|c|c|c|c|}
  % after \\: \hline or \cline{col1-col2} \cline{col3-col4} ...
  \hline
  \textbf{Parámetro} & \textbf{Tamaño Muestral} & \textbf{Varianzas} & \textbf{Límite Inferior} & \textbf{Límite Superior}\\
  \hline\hline
  \rule[-.4cm]{0cm}{1cm} $\mu$ & $n\geq 30$ & Conocida $\sigma^2$ &  $\displaystyle \bar{x}-z_{\alpha/2}\frac{\sigma}{\sqrt{n}}$ &
   $\displaystyle \bar{x}+z_{\alpha/2}\frac{\sigma}{\sqrt{n}}$ \\ \cline{3-5}
  \rule[-.4cm]{0cm}{1cm}  & Cualquier población  & Desconocida & $\displaystyle \bar{x}-z_{\alpha/2}\frac{\hat s}{\sqrt{n}}$ &
  $\displaystyle \bar{x}+z_{\alpha/2}\frac{\hat s}{\sqrt{n}}$ \\ \cline{2-5}
  \rule[-.4cm]{0cm}{1cm} & $n<30$ & Conocida $\sigma^2$ & $\displaystyle \bar{x}-z_{\alpha/2}\frac{\sigma}{\sqrt{n}}$ &
  $\displaystyle \bar{x}+z_{\alpha/2}\frac{\sigma}{\sqrt{n}}$ \\ \cline{3-5}
  \rule[-.4cm]{0cm}{1cm}& Población Normal & Desconocida & $\displaystyle \bar{x}-t^{n-1}_{\alpha/2}\frac{\hat s}{\sqrt{n}}$ &
  $\displaystyle \bar{x}+t^{n-1}_{\alpha/2}\frac{\hat s}{\sqrt{n}}$ \\ \hline
  \rule[-.4cm]{0cm}{1cm} $\sigma^2$ & Población Normal & & $\displaystyle \frac{ns^2}{\chi^{n-1}_{1-\alpha/2}}$ &
  $\displaystyle \frac{ns^2}{\chi^{n-1}_{\alpha/2}}$ \\ \hline
  \rule[-.4cm]{0cm}{1.1cm} $p$ & $n\hat{p}>5$ y $n(1-\hat{p})>5$ & &
  $\displaystyle \hat{p}-z_{\alpha/2}\sqrt{\frac{\hat{p}(1-\hat{p})}{n}}$ &
  $\displaystyle \hat{p}+z_{\alpha/2}\sqrt{\frac{\hat{p}(1-\hat{p})}{n}}$ \\ \hline
  \rule[-.5cm]{0cm}{1.3cm}$\mu_1-\mu_2$ & $n_1\geq 30$ y $n_2\geq 30$  & Conocidas $\sigma_1^2$ y $\sigma_2^2$ &
  $\displaystyle \bar{x}_{1}-\bar{x}_{2} -
  z_{\alpha/2}\sqrt{\frac{\sigma^2_{1}}{n_{1}}+\frac{\sigma^2_{2}}{n_{2}}}$ &
  $\displaystyle \bar{x}_{1}-\bar{x}_{2} +
  z_{\alpha/2}\sqrt{\frac{\sigma^2_{1}}{n_{1}}+\frac{\sigma^2_{2}}{n_{2}}}$ \\
  \cline{3-5}
  \rule[-.5cm]{0cm}{1.3cm} & Cualquier población & Desconocidas &
  $\displaystyle \bar{x}_{1}-\bar{x}_{2} -
  z_{\alpha/2}\sqrt{\frac{\hat s^2_1}{n_{1}}+\frac{\hat s^2_2}{n_{2}}}$ &
  $\displaystyle \bar{x}_{1}-\bar{x}_{2} +
  z_{\alpha/2}\sqrt{\frac{\hat s^2_1}{n_{1}}+\frac{\hat s^2_2}{n_{2}}}$ \\
  \cline{2-5}
  \rule[-.5cm]{0cm}{1.4cm}& $n_1< 30$ o $n_2< 30$  & Conocidas $\sigma_1^2$ y $\sigma_2^2$ &
  $\displaystyle \bar{x}_{1}-\bar{x}_{2} -
  z_{\alpha/2}\sqrt{\frac{\sigma^2_{1}}{n_{1}}+\frac{\sigma^2_{2}}{n_{2}}}$ &
  $\displaystyle \bar{x}_{1}-\bar{x}_{2} +
  z_{\alpha/2}\sqrt{\frac{\sigma^2_{1}}{n_{1}}+\frac{\sigma^2_{2}}{n_{2}}}$ \\ \cline{3-5}
  \rule[-.5cm]{0cm}{1.2cm} & Población Normal & Desconocidas e iguales &
  $\displaystyle \bar{x}_{1}-\bar{x}_{2} -
  t^{n_1+n_2-2}_{\alpha/2}\;\hat s_p\sqrt{\frac{1}{n_{1}}+\frac{1}{n_{2}}}$ &
  $\displaystyle \bar{x}_{1}-\bar{x}_{2} +
  t^{n_1+n_2-2}_{\alpha/2}\;\hat s_p\sqrt{\frac{1}{n_{1}}+\frac{1}{n_{2}}}$ \\ \cline{3-5}
  \rule[-.5cm]{0cm}{1.3cm} &  & Desconocidas y diferentes &
  $\displaystyle \bar{x}_{1}-\bar{x}_{2} -
  t^v_{\alpha/2}\sqrt{\frac{\hat s^2_1}{n_{1}}+\frac{\hat s^2_2}{n_{2}}}$ &
  $\displaystyle \bar{x}_{1}-\bar{x}_{2} +
  t^v_{\alpha/2}\sqrt{\frac{\hat s^2_1}{n_{1}}+\frac{\hat s^2_2}{n_{2}}}$ \\ \hline
  \rule[-.5cm]{0cm}{1.2cm} $\displaystyle \frac{\sigma_1^2}{\sigma_2^2}$ & Población Normal & &
  $\displaystyle \frac{\hat s^2_1}{\hat s^2_2}F^{n_{2}-1,n_{1}-1}_{\alpha/2}$ &
  $\displaystyle \frac{\hat s^2_1}{\hat s^2_2}F^{n_{2}-1,n_{1}-1}_{1-\alpha/2}$ \\
  \hline
  \rule[-.5cm]{0cm}{1.3cm}  $p_1-p_2$ & $n\hat{p}_i>5$ y $n(1-\hat{p}_i)>5$ & &
  $\displaystyle \hat{p}_{1}-\hat{p}_{2}-z_{\alpha/2}
  \sqrt{\frac{\hat{p}_{1}(1-\hat{p}_{1})}{n_{1}}+
  \frac{\hat{p}_{2}(1-\hat{p}_{2})}{n_{2}}}$ &
  $\displaystyle \hat{p}_{1}-\hat{p}_{2}+z_{\alpha/2}
  \sqrt{\frac{\hat{p}_{1}(1-\hat{p}_{1})}{n_{1}}+
  \frac{\hat{p}_{2}(1-\hat{p}_{2})}{n_{2}}}$ \\ \hline
\end{tabular}
\end{center}
\newpage
\section*{Notación}
$n$ es el tamaño muestral.\\
$\mu$ es la media poblacional.\\
$\sigma$ es la desviación típica de la población.\\
$p$ es la proporción de individuos que presentan el atributo estudiado en la población.\\
$\bar{x}$ es la media muestral.\\
$s$ es la desviación típica muestral.\\
$\hat s$ es la cuasidesviación típica muestral.\\
$\hat{p}$ es la proporción de individuos que presentan el atributo estudiado en la muestra.\\
$\displaystyle \hat s^2_p=\frac{(n_{1}-1)\hat s^2_1+(n_{2}-1)\hat s^2_2}{n_{1}+n_{2}-2}=
\frac{n_{1}s^2_{1}+n_{2}s^2_{2}}{n_{1}+n_{2}-2}$ es la cuasivarianza ponderada.\\
$v=\frac{\left(\frac{\hat s^2_1}{n_{1}}+
          \frac{\hat s^2_2}{n_{2}}\right)^2}
         {\frac{\left(\frac{\hat s^2_1}{n_{1}}\right)^2}{n_{1}+1}+
          \frac{\left(\frac{\hat s^2_2}{n_{2}}\right)^2}{n_{2}+1}} -2$, son los
          grados de libertad de la $t$ de Student en el caso de varianzas
          diferentes.\\
$z_{\alpha/2}$ es el valor de la normal estándar que deja acumulada
una probabilidad $1-\alpha/2$.\\
$t^{n-1}_{\alpha/2}$ es el valor de una $t$ de student de $n-1$
grados de libertad que deja acumulada una probabilidad
$1-\alpha/2$.\\
$\chi^{n-1}_{\alpha/2}$ es el valor de una ji-cuadrado con $n-1$
grados de libertad que deja acumulada una probabilidad $1-\alpha/2$.\\
$\chi(n-1)_{1-\alpha/2}$ es el valor de una ji-cuadrado con $n-1$
grados de libertad que deja acumulada una probabilidad $\alpha/2$.\\
$F^{n_{1}-1,n_{2}-1}_{\alpha/2}$ es el valor de una $F$ de
Fisher-Snedecor de $n_1-1$ y $n_2-1$ grados de libertad que deja acumulada una
probabilidad $1-\alpha/2$.\\
$F^{n_{1}-1,n_{2}-1}_{1-\alpha/2}$ es el valor de una $F$ de
Fisher-Snedecor de $n_1-1$ y $n_2-1$ grados de libertad que deja acumulada una
probabilidad $\alpha/2$.
\end{document}
