\documentclass[a4paper,landscape]{article}
%%%%%%%%%%%%%%%%%%%%%%%%%%%%
\usepackage[spanish]{babel}
\usepackage[utf8x]{inputenc}
\usepackage{amsmath}
\usepackage{macros}
\usepackage{graphicx}
\usepackage{multirow}
\usepackage[top=2.3cm, bottom=2.3cm, left=2.54cm, right=2.54cm]{geometry}
\usepackage{fancyhdr}
\pagestyle{fancy}


\lhead{\textsc{\textcolor[rgb]{0.00,0.00,0.50}{Universidad San Pablo CEU}}} \rhead{\textsl{\textsf{\textcolor[rgb]{0.00,0.00,0.50}{Departamento
de Matemáticas}}}}
\renewcommand{\headrulewidth}{0pt}
\renewcommand{\multirowsetup}{\centering}

\begin{document}

\begin{center}
\section*{Tabla de Contrastes de Hipótesis}
\renewcommand{\arraystretch}{2}
\begin{tabular}{|c|c|c|c|c|c|c|}
  \hline
  \textbf{Contraste} & \textbf{Población} & \textbf{Tamaño Muestral} & \textbf{Varianzas} & \textbf{Estadístico de Contraste} & \textbf{Distribución bajo $\mathbf{H_0}$} & \textbf{Región de Aceptación}\\
  \hline\hline
\multirow{4}{2cm}{$H_0:\ \mu=\mu_0$\newline $H_1:\ \mu\neq \mu_0$} & \multirow{2}{2cm}{Cualquiera} & \multirow{2}{1.2cm}{$n\geq 30$} & Conocida &  $Z=\dfrac{\bar{x}-\mu_0}{\sigma/\sqrt{n}}$ & $N(0,1)$ & $-z_{\alpha/2}< Z < z_{\alpha/2}$\\ \cline{4-7}
& & & Desconocida &  $Z=\dfrac{\bar{x}-\mu_0}{\hat s/\sqrt{n}}$ & $N(0,1)$ & $-z_{\alpha/2}< Z < z_{\alpha/2}$\\ \cline{2-7}
& \multirow{2}{2cm}{Normal} & \multirow{2}{1.2cm}{$n< 30$} & Conocida &  $Z=\dfrac{\bar{x}-\mu_0}{\sigma/\sqrt{n}}$ & $N(0,1)$ & $-z_{\alpha/2}< Z < z_{\alpha/2}$\\ \cline{4-7}
& & & Desconocida &  $Z=\dfrac{\bar{x}-\mu_0}{\hat s/\sqrt{n}}$ & $T(n-1)$ & $-t^{n-1}_{\alpha/2}< T < t^{n-1}_{\alpha/2}$ \\ \hline  
\parbox{2.2cm}{$H_0:\ \sigma^2=\sigma_0^2$\newline $H_1:\ \sigma^2\neq \sigma_0^2$} & Normal & & & $J=\dfrac{ns^2}{\sigma _0 ^2}$ & $\chi^2(n-1)$ & $\chi^{n-1}_{\alpha/2}< J < \chi^{n-1}_{1-\alpha/2}$\\ \hline
\parbox{2cm}{$H_0:\ p=p_0$\newline $H_1:\ p\neq p_0$} & Binomial & \parbox{2.5cm}{$n\hat{p}>5$ y\newline $n(1-\hat{p})>5$} & & $Z=\dfrac{\hat{p} - p_0 }{\sqrt{\frac{p_0(1-p_0)}{n}}}$ & $N(0,1)$ & $-z_{\alpha/2}< Z < z_{\alpha/2}$\\ \hline 
\multirow{5}{2.1cm}{$H_0:\ \mu_1=\mu_2$\newline $H_1:\ \mu_1\neq \mu_2$} & \multirow{2}{2cm}{Cualquiera} & \multirow{2}{2cm}{$n\geq 30$} & Conocidas & $Z=\dfrac{(\bar{x}_1-\bar{x}_2)}{\sqrt{\frac{\sigma_1 ^2}{n_1 }+ \frac{\sigma_2^2}{n_2 }}}$ & $N(0,1)$ & $-z_{\alpha/2}< Z < z_{\alpha/2}$\\ \cline{4-7}
& & & Desconocidas & $Z=\dfrac{(\bar{x}_1-\bar{x}_2)}{\sqrt{\frac{\hat s^2_1}{n_1}+ \frac{\hat s^2_2}{n_2 }}}$ & $N(0,1)$ & $-z_{\alpha/2}< Z < z_{\alpha/2}$\\ \cline{2-7}
& \multirow{3}{2cm}{Normales} & \multirow{3}{2cm}{$n<30$} & Conocidas & $Z=\dfrac{(\bar{x}_1-\bar{x}_2)}{\sqrt{\frac{\sigma_1 ^2}{n_1 }+ \frac{\sigma_2^2}{n_2 }}}$ & $N(0,1)$ & $-z_{\alpha/2}< Z < z_{\alpha/2}$ \\ \cline{4-7}
& & & \parbox{2cm}{Desconocidas\newline e iguales} & $T=\dfrac{(\bar{x}_1-\bar{x}_2)}{\hat s_p\sqrt{\frac{1}{n_1 }+ \frac{1}{n_2 }}}$ & $T(n_1+n_2-2)$ & $-t^{n_1+n_2-2}_{\alpha/2}< T < t^{n_1+n_2-2}_{\alpha/2}$ \\ \cline{4-7}
& & & \parbox{2cm}{Desconocidas\newline y diferentes} & $T=\dfrac{(\bar{x}_1-\bar{x}_2)}{\sqrt{\frac{\hat s^2_1}{n_1}+ \frac{\hat s^2_2}{n_2}}}$ & $T(\upsilon)$ & $-t^{\upsilon}_{\alpha/2}< T < t^{\upsilon}_{\alpha/2}$\\ \hline
\parbox{2.2cm}{$H_0:\ \sigma_1^2=\sigma_2^2$\newline $H_1:\ \sigma_1^2\neq \sigma_2^2$} & Normales &  &  & $F=\dfrac{\hat s^2_1}{\hat s^2_2}$ & $F(n_1-1,n_2-1)$ & $F^{n_1-1,n_2-1}_{\alpha/2}< F < F^{n_1-1,n_2-1}_{1-\alpha/2}$\\ \hline
\parbox{2cm}{$H_0:\ p_1=p_2$\newline $H_1:\ p_1\neq p_2$}& Binomiales & \parbox{2.5cm}{$n\hat{p_i}>5$ y\newline $n(1-\hat{p_i})>5$} & & $Z=\dfrac{\hat{p}_1-\hat{p_2}}{\sqrt {\frac{\hat{p_1}(1-\hat{p_1})}{n_1}+\frac{\hat{p_2}(1-\hat{p_2})}{n_2}}}$ & $N(0,1)$ & $-z_{\alpha/2}< Z < z_{\alpha/2}$\\ \hline

  

\end{tabular}
\end{center}
\newpage
\section*{Notación}
$n$ es el tamaño muestral.\\
$\mu$ es la media poblacional.\\
$\sigma$ es la desviación típica de la población.\\
$p$ es la proporción de individuos que presentan el atributo estudiado en la población.\\
$\bar{x}$ es la media muestral.\\
$s$ es la desviación típica muestral.\\
$\hat s$ es la cuasidesviación típica muestral.\\
$\hat{p}$ es la proporción de individuos que presentan el atributo estudiado en la muestra.\\
$\displaystyle \hat s^2_{p}=\frac{(n_{1}-1)\hat s^2_1+(n_{2}-1)\hat s^2_2}{n_{1}+n_{2}-2}=
\frac{n_{1}s^2_{1}+n_{2}s^2_{2}}{n_{1}+n_{2}-2}$ es la cuasivarianza ponderada.\\
$v=\frac{\left(\frac{\hat s^2_1}{n_{1}}+
          \frac{\hat s^2_2}{n_{2}}\right)^2}
         {\frac{\left(\frac{\hat s^2_1}{n_{1}}\right)^2}{n_{1}+1}+
          \frac{\left(\frac{\hat s^2_2}{n_{2}}\right)^2}{n_{2}+1}} -2$, son los
          grados de libertad de la $t$ de Student en el caso de varianzas
          diferentes.\\
$N(0,1)$ es la distribución normal estndar.\\
$T(n-1)$ es la distribución T de student de $n-1$ grados de libertad.\\
$\chi(n-1)$ es la distribución Chi-cuadrado de $n-1$ grados de libertad.\\
$F(n_{1}-1,n_{2}-1)$ es la distribución $F$ de Fisher-Snedecor de $n_1-1$ y $n_2-1$ grados de libertad.\\
\end{document}
