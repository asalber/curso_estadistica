\documentclass[a4paper]{article}
%%%%%%%%%%%%%%%%%%%%%%%%%%%%
\usepackage[spanish]{babel}
\usepackage[utf8]{inputenc}
\usepackage{type1cm}
\usepackage{amsmath}
\usepackage{multicol}
\usepackage{color}
\usepackage{macros}
\usepackage{graphicx}
%\usepackage[usenames,dvipsnames]{pstricks}
\usepackage{fancybox}
%\usepackage{times}
\usepackage[top=3cm, bottom=2.5cm, left=2cm, right=2cm, headsep=1cm]{geometry}
\usepackage{fancyhdr}
\pagestyle{fancy}
\usepackage[pdfauthor={Alfredo Sánchez Alberca}, pdftitle={Fórmulas de Estadística},colorlinks=true]{hyperref}


% Lists
\usepackage[shortlabels]{enumitem} % Customize lists
\setlist{nolistsep} % Reduce spacing between bullet points and numbered lists
\setlist[description]{style=sameline,leftmargin=0cm}

\makeatletter
\let\savees@listquot\es@listquot
\def\es@listquot{\protect\savees@listquot}
\makeatletter


\lhead{\textsc{Universidad San Pablo CEU}} \rhead{\url{http://aprendeconalf.es}}
\renewcommand{\headrulewidth}{0pt}
\renewcommand{\floatpagefraction}{.8}
\renewcommand{\textfraction}{.1} 

\setlength{\columnsep}{1cm}
\columnseprule .2pt

\newlength{\mylength}
\newenvironment{marco}{
	\setlength{\fboxsep}{5pt}
	\setlength{\mylength}{\textwidth}
	\addtolength{\mylength}{-2\fboxsep}
	\addtolength{\mylength}{-2\fboxrule}
	\noindent
	\begin{Sbox}
	\begin{minipage}{\mylength}
	\setlength{\abovedisplayskip}{3pt}
	\setlength{\belowdisplayskip}{3pt}
}
{
	\end{minipage}
	\end{Sbox}
	\fbox{\TheSbox}
}

\definecolor{royalblue1}{rgb}{0.28,0.46,1}

\begin{document}
\sloppy 
\practica[royalblue1]{}{Fórmulas de Estadística}

\footnotesize
\subsection*{Estadística Descriptiva}

\begin{marco}
\begin{multicols}{2}
\begin{description}
\item [Tamaño muestral] $n$ número de individuos de la muestra.
\end{description}
\subsubsection*{Frecuencias}
\begin{description}
\item [Frecuencia absoluta] $n_i$.
\item [Frecuencia relativa] $f_i=n_i/n$.
\item [Frec. absoluta acumulada] $N_i=\sum_{k=0}^in_i$.
\item [Frec. relativa acumulada] $F_i=N_i/n$.
\end{description}
\subsubsection*{Medidas de Representatividad}
\begin{description}
\item [Media] $\overline{x}=\dfrac{\sum x_in_i}{n}$.
\item [Mediana] $me$ Es el valor que ocupa el centro de la distribución. Tiene frecuencia acumulada $N_{me}=n/2$ y frecuencia relativa acumulada $F_{me}=0.5$.
\item [Moda] $mo$ Es el valor con mayor frecuencia absoluta.
\end{description}
\subsubsection*{Medidas de Posición}
\begin{description}
\item [Cuartiles] $c_1,c_2,c_3$ dividen la muestra en 4 partes iguales. Tienen frecuencias relativas acumuladas
$F_{c_1}=0.25$, $F_{c_2}=0.5$ y $F_{c_3}=0.75$ respectivamente.
\item [Percenciltes] $p_1,p_2,\cdots,p_{99}$ dividen la distribución en 100
partes iguales. La frecuencia relativa acumulada correspondiente a el percentil
$i$ es $F_{p_i}=i/100$.
\end{description}
\subsubsection*{Medidas de Dispersión}
\begin{description}
\item [Rango intercuartílico] $RI=c_3-c_1$.
\item [Varianza] $s^2=\dfrac{\sum x_i^2n_i}{n}-\overline{x}^2$
\item [Desviación típica] $s=+\sqrt{s^2}$.
\item [Coeficiente de variación] $cv=\dfrac{s}{|\overline{x}|}$.
\end{description}
\subsubsection*{Medidas de Forma}
\begin{description}
\item [Coeficiente de asimetría] \[g_1=\frac{\sum
(x_i-\overline{x})^3f_i}{s^3}.\]
\begin{itemize}
\item $g_1=0$ distribución simétrica.
\item $g_1<0$ distribución asimétrica a la izquierda.
\item $g_1>0$ distribución asimétrica a la derecha.
\end{itemize}
\item [Coeficiente de apuntamiento] \[g_2=\frac{\sum
(x_i-\overline{x})^4f_i}{s^4}-3.\]
\begin{itemize}
\item $g_2=0$ apuntamiento normal.
\item $g_2<0$ apuntamiento menor de lo normal.
\item $g_2>0$ apuntamiento mayor de lo normal.
\end{itemize}
\end{description}

\subsubsection*{Transformaciones Lineales}
\begin{description}
\item [Propiedades de la transformación]\mbox{$y=a+bx$}
\[\overline{y}=a+b\overline{x} \qquad s_y=bs_x\]

\item [Transformación de tipificación] $y=\dfrac{x-\overline{x}}{s_x}$.
\end{description}
\end{multicols}
\end{marco}

\subsection*{Regresión y Correlación}
\begin{marco}
\begin{multicols}{2}
\subsubsection*{Regresión}
\begin{description}
\item [Covarianza] $s_{xy}=\dfrac{\sum
x_iy_jn_{ij}}{n}-\overline{x}\overline{y}$.
\item [Rectas de regresión]
\begin{align*}
\textrm{($y$ sobre $x$) }&
y=\overline{y}+\dfrac{s_{xy}}{s_x^2}(x-\overline{x}),\\
\textrm{($x$ sobre $y$) }&
x=\overline{x}+\dfrac{s_{xy}}{s_y^2}(y-\overline{y}).
\end{align*}
\item [Coeficientes de regresión]
\[
\textrm{($y$ sobre $x$) } b_{yx}=\dfrac{s_{xy}}{s_x^2}\quad \textrm{($x$ sobre
$y$) } b_{xy}=\dfrac{s_{xy}}{s_y^2}
\]
\end{description}
\subsubsection*{Correlación}
\begin{description}
\item[Coeficiente de determinación lineal] 
\[r^2=\dfrac{s_{xy}^2}{s_x^2s_y^2} \qquad 0\leq r^2\leq 1\]
\begin{itemize}
\item $r^2=0$ no hay relación lineal.
\item $r^2=1$ relación lineal perfecta.
\end{itemize}

\item[Coeficiente de correlación lineal] 
\[r=\dfrac{s_{xy}}{s_xs_y}.\qquad -1\leq r\leq 1\]
\begin{itemize}
\item $r=0$ no hay relación lineal.
\item $r=-1$ relación lineal perfecta decreciente.
\item $r=1$ relación lineal perfecta creciente.
\end{itemize}
\end{description}
\subsubsection*{Regresión no lineal}
\begin{description}
  \item[Modelo exponencial] $y=e^{a+bx}$\\
  Aplicar el logaritmo a la variable dependiente y calcular la recta $\log y = a+bx$.
  \item[Modelo logarítmico] $y=a+b\log x$\\
  Aplicar el logaritmo a la variable independiente y calcular la recta $y=a+b\log x$.
  \item[Modelo potencial] $y=ax^b$\\
  Aplicar el logaritmo a la variable dependiente e independiente y calcular la recta $\log y = a+b\log x$.
\end{description}

\subsubsection*{Relaciones entre atributos}
\begin{description}
\item[Coeficiente de correlación de spearman] Es el coeficiente de correlación lineal aplicado a los órdenes de los valores de la
variable.
\item[Coeficiente chi-cuadrado]
\[
\chi^2 = \sum \frac{\left(n_{ij}-\frac{n_{x_i}n_{y_j}}{n}\right)^2}{\frac{n_{x_i}n_{y_j}}{n}},
\]
\item[Coeficiente de contingencia]
\[
C = \sqrt{\frac{\chi^2}{\chi^2+n}} \qquad 0\leq C <1
\]
\end{description}
\end{multicols}
\end{marco}

\subsection*{Probabilidad}
\begin{marco}
\begin{multicols}{2}
\begin{description}
\item [Espacio muestral $E$] Es el conjunto de posibles resultados de un experimento.
\end{description}
\subsubsection*{Operaciones entre Sucesos}
\begin{description}
\item [Unión] $A\cup B$ Elementos de $A$ más elementos de $B$.
\item [Intersección] $A\cap B$ Elementos comunes en $A$ y $B$.
\item [Contrario] $\overline{A}$ Elementos de $E$ menos los de $A$.
\item [Diferencia] $A-B$ Elementos de $A$ menos los de $B$.\\
$A-B =A\cap \overline{B} = A -(A\cap B)$.
\item [Sucesos incompatibles] $A\cap B=\emptyset$.
\end{description}
\subsubsection*{Propiedades de las operaciones entre sucesos}
\begin{description}
\item [Conmutativa] $A\cup B=B\cup A$ y $A\cap B=B\cap A$.
\item [Asociativa] $A\cup (B\cup C)=(A\cup B)\cup C$ y $A\cup (B\cap C)=(A\cap
B)\cap C$.
\item [Elemento Neutro] $A\cup \emptyset=A$ y $A\cap E=A$.
\item [Elemento Antisimétrico] $A\cup \overline{A}=E$ y $A\cap
\overline{A}=\emptyset$.
\item [Elemento Absorbente] $A\cup E=E$ y $A\cap \emptyset=\emptyset$.
\item [Leyes de Morgan] $\overline{A\cup B}=\overline{A}\cap\overline{B}$ y $\overline{A\cap B}=\overline{A}\cup\overline{B}$
\end{description}
\subsubsection*{Probabilidad}
\begin{description}
\item [Regla de Laplace] \[P(A)=\dfrac{n_A}{n_E}\quad \left(\dfrac{\textrm{casos favorables}}{\textrm{casos
posibles}}\right).\] Sólo se aplica cuando hay equiprobabilidad.
\item [Probabilidad del contrario] \[P(\overline A)=1-P(A).\]
\item [Probabilidad de la unión] \[P(A\cup B)=P(A)+P(B)-P(A\cap B).\]
\item [Probabilidad condicionada] \[P(A/B)=\dfrac{P(A\cap B)}{P(B)}.\]
\item [Probabilidad de la intersección] \[P(A\cap B)=P(A)P(B/A).\]
\item [Sucesos independientes] $P(A/B)=P(A)$.
\item [Sistema completo de sucesos] $A_1,\ldots,A_n$ deben cumplir
\begin{itemieze}
\item $A_1\cup\cdots\cup A_n=E$.
\item $A_i\cap A_j=\emptyset$ si $i\neq j$.
\end{itemize}

\item [Teorema de la probabilidad total]
\[ P(B)=\sum_{i=1}^{n}P(A_i)P(B/A_i).\]
\item [Teorema de Bayes]
\[P(A_i/B)=\dfrac{P(A_i)P(B/A_i)}{\sum_{i=1}^{n}P(A_i)P(B/A_i)}\]
\end{description}
\end{multicols}
\end{marco}

\subsection*{Variables Aleatorias}
\begin{marco}
\begin{multicols}{2}
\begin{description}
\item [Tamaño poblacional] $N$ número de individuos de la población.
\end{description}
\subsubsection*{Variables Aleatorias Discretas}
\begin{description}
\item [Función de probabilidad] $f(x_i)=P(X=x_i)$.
\item [Función de distribución] $F(x_i)=P(X\leq x_i)$.
\item [Media o esperanza] $E[X]=\mu=\sum x_if(x_i)$.
\item [Varianza] $V[X]=\sigma^2=\sum x_i^2f(x_i)-\mu^2$.
\item [Desviación típica] $\sigma=+\sqrt{\sigma^2}$.
\end{description}
\subsubsection*{Modelos de Distribución Discretos}
\begin{description}
\item [Uniforme] $U(k)$
\[f(x)=1/k.\]
\item [Binomial] $B(n,p)$
\[f(x)=\binom{n}{x}p^x(1-p)^{n-x}.\]
\[\mu=n\cdot p\quad \sigma=\sqrt{n\cdot p\cdot (1-p)}.\]
\item [Poisson] $P(\lambda)$
\[f(x)=e^{-\lambda}\frac{\lambda^x}{x!}\]
\[\mu=\lambda\quad \sigma=\sqrt{\lambda}.\]
\end{description}
\subsubsection*{Variables Aleatorias Continuas}
\begin{description}
\item [Función de densidad] $f(x)$ Debe cumplir
\begin{itemize}
\item $f(x)\geq 0$.
\item $\int_{-\infty}^\infty f(x)\,dx=1$.
\end{itemize}
\item [Función de distribución] \[F(x)=P(X\leq x)=\int_{-\infty}^x f(x)\,dx.\]
\item [Probabilidad de un intervalo]
\[P(a\leq X\leq b)=\int_a^b f(x)\,dx=F(b)-F(a).\]
\item [Media] $E[X]=\mu=\int_{-\infty}^\infty xf(x)\,dx$.
\item [Varianza] $V[X]=\sigma^2=\int_{-\infty}^\infty x^2f(x)\,dx-\mu^2$.
\end{description}
\subsubsection*{Modelos de Distribución Continuos}
\begin{description}
\item [Uniforme] $U(a,b)$
\[f(x)=\frac{1}{b-a}.\]
\item [Normal] $N(\mu,\sigma)$.
\[
f(x)=\dfrac{1}{\sigma \sqrt{2\pi }}\ e^{-\dfrac{(x-\mu )^{2}}{2\sigma ^{2}}}
\]
\end{description}
\end{multicols}
\end{marco}
\end{document}
